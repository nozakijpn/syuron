\documentclass[11pt,a4paper]{jsarticle}
%
\usepackage{amsmath,amssymb}
\usepackage{bm}
\usepackage{graphicx}
\usepackage{ascmac}
%
\setlength{\textwidth}{\fullwidth}
\setlength{\textheight}{39\baselineskip}
\addtolength{\textheight}{\topskip}
\setlength{\voffset}{-0.5in}
\setlength{\headsep}{0.3in}
%
\newcommand{\divergence}{\mathrm{div}\,}  %ダイバージェンス
\newcommand{\grad}{\mathrm{grad}\,}  %グラディエント
\newcommand{\rot}{\mathrm{rot}\,}  %ローテーション
%
\pagestyle{myheadings}
\begin{document}
%
%
\section*{卒論、修論のタイトル打ち合わせ}
\section{卒論、修論のタイトルの要素}
\subsection{何をやるか(What)}
ニュース番組音声において、発話から抽出されるi-vectorを用いてアンカーの発話区間を検出し、音声認識を行う。

\subsection{どのようにやるか(キーアイディアは何か)(How)}
短い発話からはi-vectorを十分に抽出できない\\
 →アンカーの発話区間の検出、音声認識において、誤識別・誤認識につながる。\\

以下の手法によって擬似的に長い発話を作成し、短い発話のi-vector抽出精度の向上を目指す

\begin{itemize}
\item 発話区間の間の時間情報を考慮した発話区間の結合
\item 発話環境を考慮した発話区間の結合
\end{itemize}

\subsection{結論は何?(Answer)}
アンカーの発話区間の音声認識精度向上

\section{タイトル案}

\subsection{最良と思うタイトル}
ニュース番組の発話区間情報を考慮した話者特徴量を用いた主要話者の音声認識\\
ニュース番組における主要話者の音声認識のための同一話者の発話区間結合手法→why不明\\
ニュース番組における主要話者の音声認識のための短い発話の話者特徴量抽出手法

\subsection{タイトルには入れなかったけど重要なキーワード}
話者特徴量 , 短い発話 , 非発話区間の長さ ,アンカー , 発話区間検出\\

発話区間情報:長さ、発話環境、次の発話区間までの時間

\subsection{中間発表のタイトル}
深層学習を用いた会議音声認識における同一話者情報の利用


%
%
\end{document}
