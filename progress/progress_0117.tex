\documentclass[11pt,a4paper]{jsarticle}
%
\usepackage{amsmath,amssymb}
\usepackage{bm}
\usepackage{graphicx}
\usepackage{ascmac}
%
\setlength{\textwidth}{\fullwidth}
\setlength{\textheight}{39\baselineskip}
\addtolength{\textheight}{\topskip}
\setlength{\voffset}{-0.5in}
\setlength{\headsep}{0.3in}
%
\newcommand{\divergence}{\mathrm{div}\,}  %ダイバージェンス
\newcommand{\grad}{\mathrm{grad}\,}  %グラディエント
\newcommand{\rot}{\mathrm{rot}\,}  %ローテーション
%
\pagestyle{myheadings}
\begin{document}
%
%
\section*{タイトル決め}
\section{中間発表タイトル}
深層学習を用いた会議音声認識における同一話者情報の利用

\section{キーワード}
ニュース番組音声、インデクシング、話者特徴量抽出(i-vector)、発話区間結合、アンカー、発話区間検出、音声認識
\section{タイトル候補}
音声認識の研究→音声認識の高精度化→ニュース番組音声におけるアンカーの音声認識の高精度化→ニュース番組音声におけるアンカーの音声認識精度向上のための発話形態と発話環境を考慮した話者特徴量抽出手法の検討

\begin{enumerate}
\item ニュース番組音声を用いた発話区間の結合による話者特徴量抽出の高精度化と音声認識への効果
\item ニュース番組音声を用いた発話区間の結合による話者特徴量抽出の高精度化とアンカーの音声認識
\item ニュース番組音声におけるアンカーの音声認識精度向上のための発話の時間情報と発話環境を考慮した話者特徴量抽出手法の検討
\item ニュース番組音声における発話の時間情報と発話環境を考慮した話者特徴量抽出手法の検討
\item ニュース番組におけるアンカーの音声認識のための発話の時間情報と発話環境を考慮した話者特徴量抽出
\end{enumerate}\par

\subsection{審議の結果}
ニュース番組におけるアンカーの発話区間の音声認識のための発話の時間情報と発話環境を考慮した話者特徴量抽出

ニュース番組を用いた発話の時間情報と発話環境を考慮した話者特徴量抽出によるアンカーの発話区間の音声認識

ニュース番組を用いたアンカーの音声認識のための発話の時間情報と発話環境を考慮した話者特徴量抽出



%
%
\end{document}
