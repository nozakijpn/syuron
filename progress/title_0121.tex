\documentclass[11pt,a4paper]{jsarticle}
%
\usepackage{amsmath,amssymb}
\usepackage{bm}
\usepackage{graphicx}
\usepackage{ascmac}
%
\setlength{\textwidth}{\fullwidth}
\setlength{\textheight}{39\baselineskip}
\addtolength{\textheight}{\topskip}
\setlength{\voffset}{-0.5in}
\setlength{\headsep}{0.3in}
%
\newcommand{\divergence}{\mathrm{div}\,}  %ダイバージェンス
\newcommand{\grad}{\mathrm{grad}\,}  %グラディエント
\newcommand{\rot}{\mathrm{rot}\,}  %ローテーション
%
\pagestyle{myheadings}
\begin{document}
%
%
\section*{卒論、修論のタイトル打ち合わせ}
\section{卒論、修論のタイトルの要素}
\subsection{何をやるか(What)}
ニュース番組音声のアンカーの発話区間の音声認識

\subsection{どのようにやるか(キーアイディアは何か)(How)}
発話の時間情報と発話環境を考慮した発話区間の結合

\subsection{結論は何?(Answer)}
音声認識精度の向上

\section{タイトル案}

\subsection{最良と思うタイトル}
ニュース番組を用いたアンカーの音声認識のための発話の時間情報と発話環境を考慮した話者特徴量抽出

\subsection{タイトルには入れなかったけど重要なキーワード}
アンカーの発話区間検出

\subsection{中間発表のタイトル}
深層学習を用いた会議音声認識における同一話者情報の利用

%
%
\end{document}
