\chapter{i-vectorを用いたアンカーの発話区間検出手法および音声認識手法とその課題}

\section{i-vectorを用いたアンカーの発話区間検出手法\cite{nozaki_gakuseikai}}
\label{section:clustering}
ニュース番組では、アンカー以外にインタビューイ(インタビューの受け手)や中継の有無によって話者数が大きく異なる。そのためクラスタ数を決定した場合、クラスタ数と話者数に不一致が起こり同一アンカーの発話群検出精度が低下する場合がある。そこで,同一話者の発話データのi-vectorはベクトル空間上で局所的に分布することに着目した。アンカーの発話数は非アンカーと比較して多いことから多くのアンカーの発話が局所的に集まると考えたため、同一アンカーの発話データをより精度よく検出できると考えた。\par
そこで、2つの発話データのi-vectorのコサイン類似度が閾値$Th_{cos}$以上の場合、その2つの発話データの話者は同一話者であると仮定した。まず、全ての発話データ間のi-vectorのコサイン類似度を求める。次に、このコサイン類似度が閾値$Th_{cos}$以上となる発話データ数が最も多い発話データを同一アンカーの発話データ群$O$のセントロイドとし、閾値$Th_{cos}$以上(話者性が類似している)の全データをそのデータ群$O$の初期要素とする。\par
一方、i-vectorを抽出する発話データの発声の抑揚が大きい場合、同一話者の発話間のi-vectorであってもコサイン類似度が閾値$Th_{cos}$以下になる場合がある。そこで、発話データ$u_i(\in O)$と発話データ群$O$の距離が一定距離以内であるとき、発話データ$u_i$は発話データ群$O$の要素として追加する。\par

\section{i-vectorを用いた音声認識手法\cite{yoshimura_clustering}}
\label{section:yoshimura_pre_clustering}
学習データに含まれる話者の音響特徴、話者特徴ごとに作成した木構造話者クラスタ、音響モデルを作成し、音声認識を行なった。この木構造話者クラスタは、母音の定常状態であるHMMの中央の状態の平均と分散を用いたBhattacharyya距離によるk-means法によって作成した。クラスタの個数は、最上位のクラスタを2分割し、作成された2つのクラスタをさらに2分割した計7つのクラスタを使用する。\par
認識の際は、学習データに用いた話者のi-vectorと評価データのi-vectorのコサイン類似度を求める。求めたコサイン類似度の高い上位$n$人の学習データを全て含んでいるクラスタをその評価データを認識するクラスタとする。この手法では木構造話者クラスタの性質上、$n$人の学習データをすべて含んでいるクラスタが複数存在する可能性がある。その場合、より下層のクラスタの方が選択された話者の割合が高いため、より下層のクラスタを認識するクラスタとする。\par
以上の木構造話者クラスタにより作成された音響モデルで音声認識を行う。

\section{課題}
両手法を用いた結果アンカーの発話区間検出精度、音声認識精度が向上しており、i-vectorを用いることの有意性を示している。しかし、\ref{section:pre_cos}節でも述べた通り発話が非常に短い場合はi-vectorは話者の特徴を十分に抽出出来ていない。そのため、アンカーの発話区間検出ではクラスタリングによる話者の識別が難しく、アンカーの発話区間の誤検出に繋がる。また音声認識では、認識対象の評価データと音響モデルを作成した話者クラスタに含まれる学習データとのi-vectorのコサイン類似度を計算して音響モデルを選択しているため、認識に適切な音響モデルを選択できない可能性がある。
