\chapter{i-vectorを用いたアンカーの発話区間検出手法および音声認識手法とその課題}

\section{i-vectorを用いたアンカーの発話区間検出手法\cite{nozaki_gakuseikai}}
\label{section:clustering}
本手法では、i-vectorを用いて話者クラスタリングを作成し、クラスタに含まれる発話が多いクラスタをアンカーのクラスタとして発話区間を検出した。従来は話者クラスタの作成にk-meansが多く用いられていたが、ニュース番組ではアンカー以外にインタビューイ(インタビューの受け手)や中継の有無によって話者数が大きく異なるため、
あらかじめクラスタ数を決定する必要があるk-meansクラスタリングはクラスタ数と話者数に不一致が起こり同一アンカーの発話群検出精度が低下する場合がある。そのため、アンカーの発話数は非アンカーと比較して多いことと、ベクトル空間上で話者ごとにi-vectorが局所的に集まることに着目し、多くの発話のi-vectorが局所的に分布している部分のみをクラスタリングすることで、同一アンカーの発話データをより精度よく検出できると考えた。\par
そこで、2つの発話データのi-vectorのコサイン類似度が閾値$Th_{cos}$以上の場合、その2つの発話データの話者は同一話者であると仮定した。まず、全ての発話データ間のi-vectorのコサイン類似度を求める。次に、このコサイン類似度が閾値$Th_{cos}$以上となる発話データ数が最も多い発話データを同一アンカーの発話データ群$O$のセントロイドとし、閾値$Th_{cos}$以上(話者性が類似している)の全データをそのデータ群$O$の初期要素とする。一方、i-vectorを抽出する発話データの発声の抑揚が大きい場合、同一話者の発話間のi-vectorであってもコサイン類似度が閾値$Th_{cos}$以下になる場合がある。そこで、発話データ$u_i(\in O)$と発話データ群$O$の距離が一定距離以内であるとき、発話データ$u_i$は発話データ群$O$の要素として追加する。以上の手順を繰り返してクラスタリングを行い、クラスタに含まれる発話が一定値以下となった時、クラスタリングを終了する。\par

\section{i-vectorを用いた音声認識手法\cite{yoshimura_clustering}}
\label{section:yoshimura_pre_clustering}
学習データに含まれる話者の音響特徴を考慮して木構造話者クラスタを作成し、各話者クラスタに含まれる学習データを用いて音響モデルを学習した。この木構造話者クラスタは、母音の定常状態であるHMMの中央の状態の平均と分散を用いたBhattacharyya距離によるk-means法によって作成した。クラスタの個数は、最上位のクラスタを2分割し、作成された2つのクラスタをさらに2分割した計7つのクラスタを使用する。\par
認識の際は、学習データに用いた話者のi-vectorと評価データのi-vectorのコサイン類似度を求める。求めたコサイン類似度の高い上位$n$人の学習データを全て含んでいるクラスタを選択し、選択したクラスタに含まれる学習データで学習した音響モデルを用いて音声認識を行なった。\par

\section{課題}
\ref{section:pre_cos}節でも述べた通り発話が非常に短い場合はi-vectorは話者の特徴を十分に抽出出来ない。そのため、i-vectorを用いた話者の識別が難しく、アンカーの発話区間の誤検出に繋がる。またi-vectorのコサイン類似度を計算して音響モデルを選択しているため、音声認識に適切な音響モデルを選択できない可能性がある。
