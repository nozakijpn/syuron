\chapter{結論}
本研究では、ニュース番組音声のインデクス自動付与に向けたダイアライゼーション実現のために、i-vectorを用いたニュースアンカーの発話検出精度向上を目指した。\par

i-vectorは短い発話データから話者の識別に必要な話者の特徴を表現することが難しく、ニュースアンカーの発話検出精度の低下の原因となっているため、本研究では短い発話に着目してニュースアンカーの発話検出を高精度に行うことを目的とした。\par

本研究ではニュース番組における「発話の時間間隔」と「発話環境」の二点を考慮した手法を提案し、ニュースアンカーの発話検出精度への有効性を検証した。一つ目に、同一話者が連続で発話する場合、発話と発話の間の時間間隔が非常に短くなることに着目した。つまり、発話間の時間が非常に短いとき前後の発話が同一話者の可能性が高いと考え、発話区間の結合を行うことで擬似的に長い発話を作成した。二つ目に、スタジオにいるニュースアンカーや、屋外でインタビューを受けるインタビューイなど、様々な環境で話者が発話していることに着目した。発話者の発話環境を検出するために音源識別を行い、前後の発話の発話者の発話環境が同じである可能性が高いとき同一話者であると考え、発話区間の結合を行った。以上の2通りの手法で結合した発話区間から抽出したi-vectorを用いて、ニュースアンカーの発話検出を行った。\par

\begin{comment}
ニュースアンカーの検出精度の評価指標であるF値を従来手法と比較したところ、発話の時間間隔を考慮した手法が6.5\%、発話環境を考慮した手法が2.3\%の検出精度向上が確認された。また、両方の手法を組み合わせて実験を行ったところ、従来手法と比較して4.7\%の検出精度の向上が確認されたが、発話の時間間隔のみを考慮した場合と比較すると精度が1.8\%低下した。またいずれの手法も、ニュースアンカーの発話数が少ないニュース番組の発話検出精度のP値がニュースアンカーの発話数が多いニュース番組と比較して、10\%程度低下した。\par
\end{comment}

実験の結果、発話の時間間隔を考慮して発話区間を結合することは、ニュースアンカーの発話検出精度の向上に有効であることがわかった。これは、同一話者が連続で発話する場合、発話と発話の時間間隔が非常に短いためである。また、発話環境を考慮して発話区間を結合することも、ニュースアンカーの発話検出精度の向上に有効であることが分かった。これは、アンカーからインタビューイなど、話者が切り替わった場合に発話環境が変化する場合が多いためである。しかし、ニュースアンカーの発話中に流れるVTRに含まれる雑音や音楽が発話環境の変化として誤識別する場合があったため、マルチモーダルをはじめとした映像情報など、発話環境以外の特徴を考慮することで更なる検出精度の向上が期待される。

今後、ニュースアンカーの発話検出精度向上のためにトピック検出技術を併用することが挙げられる。本実験において、ニュースアンカーの発話数が少ない場合ニュースアンカーの発話検出精度が低下した。そこで、ニュース番組におけるトピックを検出し、トピックに関連する発話を行っている話者をニュースアンカーの可能性が高い話者として推定することで、ニュースアンカーの発話数に依存しないニュースアンカーの発話検出が可能であると考えられる。\par
また、ニュース番組のインデクシングを実現するためにはニュースアンカーの発話の検出以外に、アンカーの発話内容を推定する必要がある。現在、「発話区間」「発話者」「発話内容」の全てが未知であるとき、ニュースアンカーの発話の音声認識精度はインデックスの作成に十分とは言えない。特に、ニュースアンカーの発話の音声認識精度が低下した理由として、ニュース番組内のVTRの存在がある。本研究で用いた音声認識システムでは雑音や音楽が同時に収録されていた場合、音声認識精度が極端に低下したため、ニュース番組音声から雑音除去を行う、雑音や音楽に頑健な音声認識システムを構築するなどを行う必要があると考えられる。

\begin{comment}
ニュースアンカーの発話数によって発話の検出精度が低下した理由として、本研究で用いたニュースアンカーの発話検出手法がニュース番組内において、ニュースアンカーの発話が非常に多いことに着目した手法であるためである。このため、発話の少ないニュースアンカーの発話を十分に検出できず、検出精度が低下したと考えられる。\par
\end{comment}

\begin{comment}
本稿では、ニュース番組音声のインデクス自動付与に向けたダイアライゼーション実現のために、発話間隔と発話環境を考慮したi-vectorを用いたニュースアンカーの発話検出精度向上を目指した。\par
特定話者の識別にはi-vectorが一般的に用いられるが、短い発話からは話者の識別に必要な十分な話者の特徴を抽出できない。そのため、本稿では前後の発話区間が同一話者の発話である可能性が高いとき発話区間を結合し、長い発話を擬似的に作成した。次に、結合した発話区間からi-vectorを抽出することで短い発話から得られるi-vectorの抽出精度向上を目指した。発話区間の結合には、同一話者が連続で発話する場合間をおかずに発話すること、話者が切り替わった時に発話環境が変化することに着目し、発話と発話の時間間隔を考慮する手法と、発話者の発話環境を考慮する手法を用いた。以上の手法を用いて結合した発話区間から抽出したi-vectorを用いてニュースアンカーの発話検出を行った結果、発話検出精度が約6\%向上し、ニュースアンカーの発話検出への有意性を示した。しかし、ニュースアンカーの発話数が少ないニュース番組においてはアンカーの検出精度が著しく低下した。このため、ニュースアンカーの発話が少ない場合においても発話を検出する手法を提案する必要がある。\par
また、本研究で抽出したi-vectorを用いてニュースアンカーの音声認識を行なった。音声認識はニュースアンカーの発話区間が既知の場合と未知の場合で行い、発話区間が既知のときは音声認識精度の向上が確認できなかった。ニュースアンカーの発話区間が未知の場合、従来と比較してニュースアンカーの発話区間検出精度が向上したことが音声認識精度の向上に繋がった。\par
今後の課題として、ニュースアンカーの発話が少ない番組におけるニュースアンカーの発話検出精度の向上が必要である。これは、発話内容などi-vector以外のニュースアンカーの特徴を考慮する必要があると考えられる。また、音声認識においては、ニュース音声の背景雑音の除去、雑音に頑健な音響モデルの作成が挙げられる。ニュースアンカーはスタジオで発話しているため基本的には音声以外は入らないが、参考映像などの音が発話中に流れることがある。そのため、雑音、音楽に対して音源分離による雑音除去や、雑音や音楽が含まれた学習データを用いて音響モデルを学習することで、音声認識精度が向上すると考えられる。
\end{comment}
