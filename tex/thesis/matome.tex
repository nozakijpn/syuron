\chapter{結論}
本稿では、ニュース番組音声のインデクス自動付与に向けたダイアライゼーション実現のために、発話間隔と発話環境を考慮したi-vectorを用いたニュースアンカーの発話検出精度向上を目指した。\par
特定話者の検出、照合にはi-vectorが用いられるが、短い発話からは話者の識別に必要な話者の特徴を抽出できない。そのため、本稿では前後の発話区間が同一話者の発話である可能性が高いとき発話区間を結合し、長い発話を擬似的に作成、結合した発話区間からi-vectorを再抽出することでi-vectorの抽出精度向上を目指した。発話区間の結合には、同一話者が連続で発話する場合間髪を入れずに発話すること、話者が切り替わった時に発話環境が変化することに着目し、発話と発話の時間間隔を考慮する手法と、発話者の発話環境を考慮する手法を用いた。以上の手法を用いて結合した発話区間からi-vectorの抽出を行なった結果、ニュースアンカーの発話区間検出が約6\%向上し、ニュースアンカーの発話区間検出への有意性を示した。\par
次に、本研究で抽出したi-vectorを用いてニュースアンカーの音声認識を行なった。音声認識はニュースアンカーの発話区間が既知の場合と未知の場合で行い、発話区間が既知のときは音声認識精度の向上が確認できなかった。しかし、ニュースアンカーの発話区間が未知の場合、従来と比較してニュースアンカーの発話区間検出精度が向上したことが音声認識精度の向上に繋がった。\par
今後の課題として、ニュース音声の背景雑音の除去が挙げられる。ニュースアンカーはスタジオで発話しているため基本的には雑音が入らないが、参考映像などの音が発話中に流れることがある。そのため、雑音、音楽に対して音源分離などの処理をすることでi-vectorの抽出精度をはじめ、アンカーの発話区間検出精度、音声認識精度が向上すると考えられる。