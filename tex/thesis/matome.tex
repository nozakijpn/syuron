\chapter{結論}
本稿では、ニュース番組音声のインデクス自動付与に向けたダイアライゼーション実現のために、発話間隔と発話環境を考慮したi-vectorを用いたニュースアンカーの発話検出精度向上を目指した。\par
特定話者の識別にはi-vectorが一般的に用いられるが、短い発話からは話者の識別に必要な十分な話者の特徴を抽出できない。そのため、本稿では前後の発話区間が同一話者の発話である可能性が高いとき発話区間を結合し、長い発話を擬似的に作成した。次に、結合した発話区間からi-vectorを抽出することで短い発話から得られるi-vectorの抽出精度向上を目指した。発話区間の結合には、同一話者が連続で発話する場合間をおかずに発話すること、話者が切り替わった時に発話環境が変化することに着目し、発話と発話の時間間隔を考慮する手法と、発話者の発話環境を考慮する手法を用いた。以上の手法を用いて結合した発話区間から抽出したi-vectorを用いてニュースアンカーの発話検出を行った結果、発話検出精度が約6\%向上し、ニュースアンカーの発話検出への有意性を示した。しかし、ニュースアンカーの発話数が少ないニュース番組においてはアンカーの検出精度が著しく低下した。このため、ニュースアンカーの発話が少ない場合においても発話を検出する手法を提案する必要がある。\par
また、本研究で抽出したi-vectorを用いてニュースアンカーの音声認識を行なった。音声認識はニュースアンカーの発話区間が既知の場合と未知の場合で行い、発話区間が既知のときは音声認識精度の向上が確認できなかった。ニュースアンカーの発話区間が未知の場合、従来と比較してニュースアンカーの発話区間検出精度が向上したことが音声認識精度の向上に繋がった。\par
今後の課題として、ニュースアンカーの発話が少ない番組におけるニュースアンカーの発話検出精度の向上が必要である。これは、発話内容などi-vector以外のニュースアンカーの特徴を考慮する必要があると考えられる。また、音声認識においては、ニュース音声の背景雑音の除去、雑音に頑健な音響モデルの作成が挙げられる。ニュースアンカーはスタジオで発話しているため基本的には音声以外は入らないが、参考映像などの音が発話中に流れることがある。そのため、雑音、音楽に対して音源分離による雑音除去や、雑音や音楽が含まれた学習データを用いて音響モデルを学習することで、音声認識精度が向上すると考えられる。
