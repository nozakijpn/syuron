\chapter{序章}
\section{はじめに}
近年、通信・放送業界では地上デジタル放送の開始や、新たな高速通信規格の誕生など、通信ネットワークの急速な発達が見られる。それに伴い、誰もがテレビやパソコンだけでなくスマートフォン・タブレットなど様々なデバイスを通して手軽に膨大な量の音声・映像データを入手し、好きな時に好きな場所で視聴することが容易な時代となった。しかし、これらの情報全てが必要な情報とは限らず、ほとんどの場合取捨選択をする必要がある。ニュース番組で例えると、ニュース番組はスポーツ、経済、社会、天気予報など様々なジャンルのトピックで構成されていて、視聴者は時間の都合や興味の違いに応じて必要なトピックのみを早送りなどで視聴する。そこで、これらのニュース番組に話者や内容のインデックスの情報が付与されていれば、所望のトピックのみを視聴することができる。また、テレビ局などの管理する側も、データベースの構築が容易になるというメリットがある。しかし世の中には膨大な量のニュース番組が存在しており、それら全てに人手でインデクスを付与することは事実上不可能であるため、自動的にインデクシングすることが望まれる。\par
自動でニュース番組のインデクシングを行うためには、ニュース番組内の発話区間、発話者、発話内容の特定が必要であり、これらを推定する技術をダイアライゼーションと呼ぶ。ニュース番組には発話数が多く、ニュース番組の司会およびトピックの切り替えを行う話者としてアナウンサー(ニュースアンカー)が存在する。ニュースアンカーの発話にはトピックのキーワードが多く含まれており、インデクシングに重要な情報を持つ。つまり、アンカーの発話区間を検出し、音声認識を行うことはニュース番組のダイアライゼーションの実現に有効であると考えられる。\par
ニュースアンカーの発話を音声認識するためには、まずニュースアンカーの発話を検出する必要がある。特定話者の話者識別には話者特徴量(i-vector)が一般的に用いられている\cite{ogawa_ivector}\cite{nishi_multi}。i-vectorとは、ある発話から得られた音響特徴量を因子分解して抽出された話者固有の特徴量である。特徴抽出に因子分解を用いているため、次元を削減して特徴を表現することが可能である。近年の話者認識システムの多くはi-vectorに基づいて構築されており、この領域における最高水準となっている。しかし、これまでのi-vectorを用いた話者識別に関する研究は、事前に対象のデータに登場する話者の情報を学習し、その学習データを用いて話者を識別する手法をとっているため、発話区間、発話者の情報が未知の場合が多いニュース番組にそれらの手法を用いることは出来ない。そこで先行研究\cite{nozaki_gakuseikai}\cite{adachi_gakuseikai}では、ニュース番組音声の発話区間、発話者が未知の場合においても用いることができるニュースアンカーの発話検出手法を提案した。これらの研究でもi-vectorを用いており、ニュースアンカーの発話検出を行った結果、ニュースアンカーの発話を約70\%の精度で検出することが出来た。\par
しかし、短い発話から抽出されたi-vectorは話者の特徴を十分に表すことが難しいことが知られており\cite{panaiv}、先行研究\cite{nozaki_gakuseikai}でもニュースアンカーの発話検出精度の原因として述べられている。つまり、短い発話からi-vectorを抽出した場合、ニュースアンカーの発話検出精度が低下する可能性がある。そこで、話者識別に必要な話者の特徴を短い発話から抽出することが可能になればニュースアンカーの発話区間検出精度の向上に有効であると考えた。\par
本研究では、発話から抽出されるi-vectorに加えて、「発話の時間間隔」と「発話環境」を考慮し、前後の発話区間が同一話者の発話である可能性が高いとき発話区間を結合した。これによって、長い発話を擬似的に作成し、短い発話のi-vectorの抽出精度向上を目指した。検証の結果、結合した発話区間から抽出したi-vectorを用いることでニュースアンカーの発話区間検出精度が約6\%向上した。よって、「発話の時間間隔」と「発話環境」を考慮して発話区間を結合することで短い発話から抽出するi-vectorが話者の特徴を抽出出来たことによってニュースアンカーの発話を検出することができることを示した。

\section{論文構成}
次章以降における本論文の構成は、まず2章で、インデクシングの流れとi-vectorに関係する基本知識の説明を行う。次に3章で、本論文で着目したニュース番組における「発話の時間間隔」と「発話環境」の調査を行う。4章では提案手法によって結合した発話区間から抽出したi-vectorを用いてニュースアンカーの発話検出実験を行い、提案手法によるニュースアンカーの発話検出精度への効果を検証する。
