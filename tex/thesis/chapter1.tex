\chapter{研究背景}

\section{研究背景}
近年、通信・放送業界では地上デジタル放送の開始や、新たな高速通信規格の誕生など、通信ネットワークの急速な発達が見られる。それに伴い、誰もがテレビやパソコンだけでなくスマートフォン・タブレットなど様々なデバイスを通して手軽に膨大な量の音声・映像データを入手し、好きな時に好きな場所で視聴することが容易な時代となった。しかし、入手できる情報量が増えた分、それら全てが必要であるとは限らず、自分に必要な情報のみを手軽に取捨選択できれば便利である。映像・音声データに、話者や内容のインデックスの情報が付与されていれば、その部分だけを選択して視聴できる。しかし、世の中には膨大な量の映像・音声データが存在するため、それら全てに人手でインデックスを付与することは事実上不可能である。そこで、自動的にインデクシングすることが望まれる。\par
自動でインデクシングを行うためには、映像・音声データ内の発話区間、発話者、発話内容の特定が必要である。これらを推定する技術のことをダイアライゼーションと呼び、本研究はこの技術の実現を目指す。\par
本研究では、世の中に存在する映像・音声データの中である特定の人物に情報が集中する形式で行われるニュース番組に着目した。ニュース番組は主にアンカー(司会役のアナウンサー)を中心として、アンカーがレポーターなどに話を振りながら、ニュースが進行していく。また、ニュース番組は収録環境が良いため、研究対象としても適しており、ニュース番組で高精度にダイアライゼーションができると、同じスタイルのその他の映像・音声データにも用いることができると考えられる。そこで、本研究ではニュース番組を対象として研究を行う。

\section{研究目的}
ダイアライゼーションで推定する情報の中で、本研究では発話者の特定に着目した。本研究で対象とするニュース番組には以下の特徴がある。\newline

ニュース番組の特徴
\begin{itemize}
\item 30分程度のニュース番組の中で複数の多様な話題がある
\item 1人または複数のアンカーおよび天気予報士など複数の話者が存在する
\item 話者情報(話者数、性別、話者の声質など)および発話区間が未知である
\end{itemize}\par\par

このようなニュース番組において、ニュースの話題にインデキシングが行われていることは必要な話題の検索に重要である。ニュース番組のアンカーには以下のような特徴があり、インデキシングには重要な情報を持つと考えられる。\newline

アンカーの特徴
\begin{enumerate}
\item 発話数が多い
\item ニュース番組の司会および話題の切り替えを行う
\item ニュース番組の全体にわたって発話している
\end{enumerate}\par

このため、アンカーの発話区間のみの音声認識を行うことによって、より高精度なインデキシングが実現可能であると考えた。これまでに先行研究として音響特徴毎に発話者群を二分化していき、段階的に分類していく手法によって話者識別を行う研究が行われた[1]。しかし、この先行研究では話者の発話区間は既知であるとして行われたため人手によって発話区間を切り出す必要があった。そこで、本研究では、話者情報と発話区間が未知の場合でも用いることが可能なアンカーの発話区間抽出手法を提案する。

\section{論文構成}
次章以降における本論文の構成は、まず2章で音声認識システムの概要について説明を行う。次に3章では話者識別システムの概要としてi-vector、コサイン類似度の理論的背景の説明、および使用する音声データの概要の説明を行い、ニュース番組音声における発話間の i-vector のコサイン類似度を用いることによる効果を検証する。4章ではi-vectorを用いた単純なアンカーの発話区間抽出アルゴリズムによる発話区間抽出実験を行い、問題提起を行う。5章では本研究で提案するアルゴリズムの説明を行う。6章では提案手法を用いた発話区間抽出実験を行い、本研究における提案手法を用いることで話者情報と発話区間が未知の場合におけるアンカーの発話区間抽出への効果を検証する。7章では6章で抽出したアンカーの発話区間の音声認識を行い、どれほど単語が正確に認識されているかを検証する。8章では本研究において検証された実験の結果を元に結論を述べる。

