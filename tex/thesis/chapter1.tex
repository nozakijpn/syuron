\chapter{序章}
\section{はじめに}
通信ネットワークの急速な発達により、テレビだけでなくウェブ上でもニュース番組が配信されているが、全ての情報が視聴者にとって重要であることはなく、視聴者は膨大な量のニュースの中から必要な情報のみを取捨選択する必要がある。一般的に、ニュース番組にはスポーツ、経済、社会、天気予報など様々なジャンルのトピックで構成されており、視聴者は時間の都合や興味の違いに応じて必要なトピックのみを早送りなどをして視聴している。現在、通信・放送業界では地上デジタル放送の開始や、新たな高速通信規格の誕生などに伴い、誰もがテレビやパソコンだけでなくスマートフォン・タブレットなど様々なデバイスを通して好きな時に好きな場所で情報を得ることができるため、情報を容易に取捨選択できるシステムは極めて重要度が高い。\par

そこで、容易な情報の取捨選択のために、情報にインデクスを付与するインデクシングという技術がある。インデクスが付与されていれば、所望のトピックのみを視聴することができる。また、情報を管理する側も、データベースの構築が容易になるというメリットがある。しかし世の中には膨大な量の情報が存在しているため、それら全てに人手でインデクスを付与することは事実上不可能であるため、自動的にインデクシングすることが望まれる。\par

インデクシングを行う上での重要な構成要素として「発話区間」「発話者」「発話内容」があり、これらを推定する技術をダイアライゼーションと呼ぶ。また、インデクシングは発話内容を要約処理し、該当する発話区間に要約内容と発話者のラベルを付与することで成り立つため、各要素の推定には高い精度が必要である。\par

「発話区間」の推定するための技術をVAD(Voice Activity Detection)技術といい、先行研究\cite{yamaguchi_indexing}では音源識別によって発話区間の検出を行い、発話区間の検出精度のF値(F-measure)が0.948であった。本先行研究では、音源識別を行うためのスペクトルの傾きや変化を含めた7つのパラメータを用いており、音声データから「音声」「背景雑音」「音楽」「無音」の4種類の区間を検出することで発話区間の検出をおこなった。\par

「発話内容」の推定には音声データから音声認識を行う必要があり、先行研究\cite{yoshimura_clustering}では木構造話者クラスタを用いて深層学習を用いた音声認識を行うことで、音声認識精度であるAcc(Word Accuracy)が74.56\%から75.23\%に向上した。本先行研究では、話者の音響的な特徴量から認識対象の話者毎に話者適応を行っている。\par

「発話者」の推定には先行研究\cite{nozaki_gakuseikai}として、i-vectorを用いたニュースアンカー(アナウンサー)の発話検出手法が提案されており、発話検出精度のF値は0.634から0.707に向上した。i-vectorとは、ある発話から得られた音響特徴量を因子分解して抽出された話者固有の特徴量である。特徴抽出に因子分解を用いているため、次元を削減して特徴を表現することが可能である。近年の話者認識システムの多くはi-vectorに基づいて構築されており、この領域における最高水準となっている。しかし、短い発話から抽出されたi-vectorは話者の特徴を十分に表現することが出来ないため\cite{panaiv}、ニュースアンカーの短い発話をニュースアンカーの発話として検出することは非常に難しい。そのため、検出精度向上のためにはニュースアンカーの短い発話を高精度に検出する必要があると述べられている。\par

インデクシングを行う上でニュースアンカーの発話区間を検出することはインデクシングに有効であるため、本研究ではニュースアンカーの短い発話の検出精度向上を目指す。これは、ニュースアンカーの発話にはニュースのトピックや重要な単語に多く含まれているため、ニュースアンカーの発話のみを検出、要約処理を行うことはニュース番組のインデクシングにおいて重要な意味を持つためである。そのため、先行研究でニュースアンカーの発話として検出することが困難であった短い発話の検出精度向上のための手法を提案する。\par

ニュースアンカーの短い発話の検出精度を向上させるための手法として、発話の前後に存在する発話が同一話者である可能性が高いときに発話区間を結合することと、結合した発話区間からi-vectorの抽出を行う。これは発話区間を結合することで擬似的に長い発話を作成し、i-vectorが話者の特徴を十分に抽出するための発話を確保するためである。本手法によって抽出されたi-vectorを用いてニュースアンカーの発話区間検出を行った。\par

本研究における提案手法を用いたニュースアンカーの発話検出実験の結果、発話の検出精度のF値が0.707から0.772となり、6.5\%の向上が確認された。\par

提案手法を用いて発話区間を結合し、結合した発話区間から抽出したi-vectorを用いることは、ニュースアンカーの発話検出への有意性を示した。

\section{論文構成}
次章以降における本論文の構成は、まず2章で、インデクシングの流れとi-vectorに関係する基本知識の説明を行う。次に3章で、本論文で着目したニュース番組における「発話の時間間隔」と「発話環境」の調査を行う。4章では提案手法によって結合した発話区間から抽出したi-vectorを用いてニュースアンカーの発話検出実験を行い、提案手法によるニュースアンカーの発話検出精度への効果を検証する。

\begin{comment}
\chapter{aa}
近年、通信・放送業界では地上デジタル放送の開始や、新たな高速通信規格の誕生など、通信ネットワークの急速な発達が見られる。それに伴い、誰もがテレビやパソコンだけでなくスマートフォン・タブレットなど様々なデバイスを通して手軽に膨大な量の音声・映像データを入手し、好きな時に好きな場所で視聴することが容易な時代となった。しかし、これらの情報全てが必要な情報とは限らず、ほとんどの場合取捨選択をする必要がある。ニュース番組で例えると、ニュース番組はスポーツ、経済、社会、天気予報など様々なジャンルのトピックで構成されていて、視聴者は時間の都合や興味の違いに応じて必要なトピックのみを早送りなどで視聴する。そこで、これらのニュース番組に話者や内容のインデックスの情報が付与されていれば、所望のトピックのみを視聴することができる。また、テレビ局などの管理する側も、データベースの構築が容易になるというメリットがある。しかし世の中には膨大な量のニュース番組が存在しており、それら全てに人手でインデクスを付与することは事実上不可能であるため、自動的にインデクシングすることが望まれる。\par
自動でニュース番組のインデクシングを行うためには、ニュース番組内の発話区間、発話者、発話内容の特定が必要であり、これらを推定する技術をダイアライゼーションと呼ぶ。ニュース番組には発話数が多く、ニュース番組の司会およびトピックの切り替えを行う話者としてアナウンサー(ニュースアンカー)が存在する。ニュースアンカーの発話にはトピックのキーワードが多く含まれており、インデクシングに重要な情報を持つ。つまり、アンカーの発話区間を検出し、音声認識を行うことはニュース番組のダイアライゼーションの実現に有効であると考えられる。\par
ニュースアンカーの発話を音声認識するためには、まずニュースアンカーの発話を検出する必要がある。特定話者の話者識別には話者特徴量(i-vector)が一般的に用いられている\cite{ogawa_ivector}\cite{nishi_multi}。i-vectorとは、ある発話から得られた音響特徴量を因子分解して抽出された話者固有の特徴量である。特徴抽出に因子分解を用いているため、次元を削減して特徴を表現することが可能である。近年の話者認識システムの多くはi-vectorに基づいて構築されており、この領域における最高水準となっている。しかし、これまでのi-vectorを用いた話者識別に関する研究は、事前に対象のデータに登場する話者の情報を学習し、その学習データを用いて話者を識別する手法をとっているため、発話区間、発話者の情報が未知の場合が多いニュース番組にそれらの手法を用いることは出来ない。そこで先行研究\cite{nozaki_gakuseikai}\cite{adachi_gakuseikai}では、ニュース番組音声の発話区間、発話者が未知の場合においても用いることができるニュースアンカーの発話検出手法を提案した。これらの研究でもi-vectorを用いており、ニュースアンカーの発話検出を行った結果、ニュースアンカーの発話を約70\%の精度で検出することが出来た。\par
しかし、短い発話から抽出されたi-vectorは話者の特徴を十分に表すことが難しいことが知られており\cite{panaiv}、先行研究\cite{nozaki_gakuseikai}でもニュースアンカーの発話検出精度の原因として述べられている。つまり、短い発話からi-vectorを抽出した場合、ニュースアンカーの発話検出精度が低下する可能性がある。そこで、話者識別に必要な話者の特徴を短い発話から抽出することが可能になればニュースアンカーの発話区間検出精度の向上に有効であると考えた。\par
本研究では、発話から抽出されるi-vectorに加えて、「発話の時間間隔」と「発話環境」を考慮し、前後の発話区間が同一話者の発話である可能性が高いとき発話区間を結合した。これによって、長い発話を擬似的に作成し、短い発話のi-vectorの抽出精度向上を目指した。検証の結果、結合した発話区間から抽出したi-vectorを用いることでニュースアンカーの発話区間検出精度が約6\%向上した。よって、「発話の時間間隔」と「発話環境」を考慮して発話区間を結合することで短い発話から抽出するi-vectorが話者の特徴を抽出出来たことによってニュースアンカーの発話を検出することができることを示した。
\end{comment}

