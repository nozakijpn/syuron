\chapter{序章}
\section{はじめに}
近年、通信・放送業界では地上デジタル放送の開始や、新たな高速通信規格の誕生など、通信ネットワークの急速な発達が見られる。それに伴い、誰もがテレビやパソコンだけでなくスマートフォン・タブレットなど様々なデバイスを通して手軽に膨大な量の音声・映像データを入手し、好きな時に好きな場所で視聴することが容易な時代となった。しかし、これらの情報全てが必要な情報とは限らず、ほとんどの場合取捨選択をする必要がある。ニュース番組で例えると、ニュース番組はスポーツ、経済、社会、天気予報など様々なジャンルのトピックで構成されていて、視聴者は時間の都合や興味の違いに応じて必要なトピックのみを早送りなどで視聴する。そこで、これらのニュース番組に話者や内容のインデックスの情報が付与されていれば、所望のトピックのみを視聴することができる。また、テレビ局などの管理する側も、データベースの構築が容易になるというメリットがある。しかし世の中には膨大な量のニュース番組が存在しており、それら全てに人手でインデクスを付与することは事実上不可能であるため、自動的にインデクシングすることが望まれる。\par
自動でニュース番組のインデクシングを行うためには、ニュース番組内の発話区間、発話者、発話内容の特定が必要であり、これらを推定する技術をダイアライゼーションと呼ぶ。ニュース番組にはアナウンサー(アンカー)が存在し、発話数が多く、ニュース番組の司会およびトピックの切り替えを行う。このため、アンカーの発話にはトピックのキーワードが多く含まれており、インデクシングに重要な情報を持つ。つまり、アンカーの発話区間を検出し、音声認識を行うことはニュース番組のダイアライゼーションの実現に有効であると考えられる。\par
アンカーの発話区間を音声認識するためには、まずアンカーの発話区間を検出する必要がある。特定話者の発話検出、話者識別には話者特徴量(i-vector)が一般的に用いられている\cite{ogawa_ivector}\cite{nishi_multi}。i-vectorとは、GMMスーパーベクトルを因子分解して得られる特徴量で、話者分類における精度が高いことが知られている。先行研究\cite{nozaki_gakuseikai}ではi-vectorを用いることは、アンカーの発話区間検出に効果的であることを示している。\par
しかし、短い発話から抽出されたi-vectorは話者の特徴を十分に表すことが難しいために話者の識別性能が低下することが示されている\cite{panaiv}。つまり、短い発話からi-vectorを抽出した場合でも話者の特徴を十分に表すことができれば、更なるアンカーの発話区間検出精度の向上が見込めると考えた。\par
そこで本研究では、発話から抽出されるi-vectorに加えて、「発話の時間間隔」と「発話環境」を考慮し、前後の発話区間が同一話者の発話である可能性が高いとき発話区間を結合した。これによって、長い発話を擬似的に作成し、短い発話のi-vectorの抽出精度向上を目指した。検証の結果、結合した発話区間から抽出したi-vectorを用いることでアンカーの発話区間検出精度が約6\%向上した。よって、「発話の時間間隔」と「発話環境」を考慮して発話区間を結合することで短い発話が少なくなり、発話から抽出するi-vectorが話者の特徴を表したことによってアンカーの発話区間を高精度に検出することができることを示した。

\section{論文構成}
