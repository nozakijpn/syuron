\chapter{研究背景}

\section{研究背景}
近年、通信・放送業界では地上デジタル放送の開始や、新たな高速通信規格の誕生など、通信ネットワークの急速な発達が見られる。それに伴い、誰もがテレビやパソコンだけでなくスマートフォン・タブレットなど様々なデバイスを通して手軽に膨大な量の音声・映像データを入手し、好きな時に好きな場所で視聴することが容易な時代となった。そこで、映像・音声データに話者や内容のインデックスの情報が付与されていれば、必要な部分のみを容易に検索、視聴ができる。しかし、世の中には膨大な量の映像・音声データが存在するため、それら全てに人手でインデクスを付与することは事実上不可能である。そこで、自動的にインデクシングすることが望まれる。\par
自動でインデクシングを行うためには、映像・音声データ内の発話区間、発話者、発話内容の特定が必要である。これらを推定する技術のことをダイアライゼーションと呼び、本稿はこの技術の実現を目指す。\par
本稿では世の中に存在する映像・音声データの中で、ある特定の人物に情報が集中する形式で行われるニュース番組に着目した。

\vspace{0.2in}\noindent{\textbf{\underline{ニュース番組の特徴}}}
\begin{itemize}
\item 30分程度のニュース番組の中で複数の多様な話題がある
\item 1人または複数のアンカーおよび天気予報士など複数の話者が存在する
\item 話者情報(話者数、性別、話者の声質など)および発話区間が未知である
\item 対話が少ない
\item アンカーは連続で発話することが多い
\end{itemize}

\noindent このようなニュース番組において、ニュースの話題にインデクシングが行われていることは必要な話題の検索に重要である。ここで、アンカーとは司会役のアナウンサーのことであり、ニュース番組は主にアンカーを中心として進行する。アンカーの特徴は以下の通りである。

\vspace{0.2in}\noindent{\textbf{\underline{アンカーの特徴}}}
\begin{itemize}
\item 発話数が多い
\item ニュース番組の司会および話題の切り替えを行う
\item ニュース番組の全体にわたって発話している
\end{itemize}

\noindent このため、ニュース番組ではアンカーの発話に情報が集中しており、インデクシングに重要な情報を持つ。つまり、アンカーの発話区間を検出、音声認識はニュース番組のダイアライゼーションの実現に有効であると考えられる。

\section{研究目的}
特定話者の発話検出には話者特徴量(i-vector)が一般的に用いられている。また、音声認識においても認識対象の話者ごとに音声認識システムを適応させることで音声認識精度の向上が確認されている。しかし、短い発話からは話者の識別に必要なi-vectorを抽出できないことが確認されており\cite{panaiv}、アンカーの発話の誤検出、音声認識精度の低下につながる可能性がある。そこで、短い発話のi-vectorの抽出精度を向上することによって、アンカーの発話区間検出精度、音声認識精度の向上が見込めると考えた。\par
本稿では、前後の発話区間が同一話者の発話である可能性が高いとき発話区間を結合し、長い発話を擬似的に作成した。これによって短い発話のi-vectorの抽出精度向上を目指し、アンカーの発話区間検出と音声認識への有意性を検証した。検証を行なった結果、従来と比較してアンカーの発話区間検出精度が9\%、音声認識精度がhogehoge\%の向上を実現した。これにより、i-vectorの抽出において発話区間を結合し擬似的に長い発話を作成することにより、アンカーの発話区間検出、音声認識への有意性を示した。

\section{論文構成}
次章以降における本論文の構成は、まず2章で本実験で用いるシステム、アルゴリズムの概要について説明を行う。次に3章では、i-vectorの性質と評価対象であるニュース番組音声の調査を行う。4章ではi-vectorを用いたアンカーの発話検出手法と音声認識の先行研究について述べ、問題提起を行う。5章では本研究で提案するアルゴリズムの説明を行う。6章では提案手法を用いた発話区間の結合、i-vectorを再抽出を行い、アンカーの発話区間検出精度、音声認識精度への効果を検証する。7章では本研究において検証された実験の結果を元に結論を述べる。

