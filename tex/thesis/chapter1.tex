\chapter{はじめに}

\section{研究背景}
近年、通信・放送業界では地上デジタル放送の開始や、新たな高速通信規格の誕生など、通信ネットワークの急速な発達が見られる。それに伴い、誰もがテレビやパソコンだけでなくスマートフォン・タブレットなど様々なデバイスを通して手軽に膨大な量の音声・映像データを入手し、好きな時に好きな場所で視聴することが容易な時代となった。しかし、これらの情報全てが必要な情報とは限らず、ほとんどの場合取捨選択をする必要がある。\par
ニュース番組で例えると、ニュース番組はスポーツ、経済、社会、天気予報など様々なジャンルのトピックで構成されていて、視聴者は時間の都合や興味の違いに応じて必要なトピックのみを早送りなどで視聴するだろう。そこで、これらのニュース番組に、話者や内容のインデックスの情報が付与されていれば、所望のトピックのみを視聴することができる。また、テレビ局などの管理する側も、データベースの構築が容易になるというメリットがある。しかし世の中には膨大な量のニュース番組が存在しており、それら全てに人手でインデクスを付与することは事実上不可能であるため、自動的にインデクシングすることが望まれる。\par
自動でインデクシングを行うためには、ニュース番組内の発話区間、発話者、発話内容の特定が必要であり、これらを推定する技術をダイアライゼーションと呼ぶ。ここで、ニュース番組には主に司会進行を行うアナウンサー(アンカー)が存在する。アンカーは発話数が多く、ニュース番組の司会およびトピックの切り替えを行う。このため、ニュース番組ではアンカーの発話にはトピックのキーワードが多く含まれており、インデクシングに重要な情報を持つ。つまり、アンカーの発話区間を検出し、音声認識を行うことはニュース番組のダイアライゼーションの実現に有効であると考えられる。

\section{研究目的}
特定話者の発話検出には話者特徴量(i-vector)が一般的に用いられている。また、音声認識においても認識対象の話者ごとに音声認識システムを適応させることで音声認識精度の向上が確認されている。しかし、短い発話からは抽出されたi-vectorは話者の特徴を十分に表すことができない\cite{panaiv}ことが確認されている。そこで、短い発話から抽出したi-vectorであっても話者の特徴を十分に表すことができれば、更なるアンカーの発話区間検出精度、音声認識精度の向上が見込めると考えた。\par
本稿では、前後の発話区間が同一話者の発話である可能性が高いとき発話区間を結合し、長い発話を擬似的に作成した。これによって短い発話のi-vectorの抽出精度向上を目指す。また、抽出したi-vectorを用いてアンカーの発話区間検出と音声認識への有意性を検証した。検証を行なった結果、従来と比較してアンカーの発話区間検出精度が6\%、音声認識精度がhogehoge\%の向上を実現した。これにより、i-vectorの抽出において発話区間を結合し擬似的に長い発話を作成することにより、アンカーの発話区間検出、音声認識への有意性を示した。

\section{論文構成}
次章以降における本論文の構成は、まず2章で本実験で用いるシステム、アルゴリズムの概要について説明を行う。次に3章では、i-vectorの性質と評価対象であるニュース番組音声の調査を行う。4章ではi-vectorを用いたアンカーの発話検出手法と音声認識の先行研究について述べ、問題提起を行う。5章では本研究で提案するアルゴリズムの説明を行う。6章では提案手法を用いた発話区間の結合とi-vectorの抽出を行い、アンカーの発話区間検出精度、音声認識精度への効果を検証する。7章では本研究において検証された実験の結論を述べる。

