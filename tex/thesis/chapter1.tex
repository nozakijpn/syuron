\chapter{序章}
\section{研究背景}
近年、通信・放送業界では地上デジタル放送の開始や、新たな高速通信規格の誕生など、通信ネットワークの急速な発達が見られる。それに伴い、誰もがテレビやパソコンだけでなくスマートフォン・タブレットなど様々なデバイスを通して手軽に膨大な量の音声・映像データを入手し、好きな時に好きな場所で視聴することが容易な時代となった。しかし、これらの情報全てが必要な情報とは限らず、ほとんどの場合取捨選択をする必要がある。\par
ニュース番組で例えると、ニュース番組はスポーツ、経済、社会、天気予報など様々なジャンルのトピックで構成されていて、視聴者は時間の都合や興味の違いに応じて必要なトピックのみを早送りなどで視聴する。そこで、これらのニュース番組に話者や内容のインデックスの情報が付与されていれば、所望のトピックのみを視聴することができる。また、テレビ局などの管理する側も、データベースの構築が容易になるというメリットがある。しかし世の中には膨大な量のニュース番組が存在しており、それら全てに人手でインデクスを付与することは事実上不可能であるため、自動的にインデクシングすることが望まれる。自動でインデクシングを行うためには、ニュース番組内の発話区間、発話者、発話内容の特定が必要であり、これらを推定する技術をダイアライゼーションと呼ぶ。ここで、ニュース番組には主に司会進行を行うアナウンサー(アンカー)が存在する。アンカーは発話数が多く、ニュース番組の司会およびトピックの切り替えを行う。このため、アンカーの発話にはトピックのキーワードが多く含まれており、インデクシングに重要な情報を持つ。つまり、アンカーの発話区間を検出し、音声認識を行うことはニュース番組のダイアライゼーションの実現に有効であると考えられる。アンカーの発話区間を音声認識するためには、まずアンカーの発話区間を検出する必要がある。特定話者の発話検出には話者特徴量(i-vector)が一般的に用いられている\cite{ogawa_ivector}。また、音声認識においてもi-vectorを用いて認識対象の話者ごとに音声認識システムを適応させることで音声認識精度の向上が確認されている。しかし、短い発話からは抽出されたi-vectorは話者の特徴を十分に表すことができない\cite{panaiv}ことが確認されている。そこで、短い発話から抽出したi-vectorであっても話者の特徴を十分に表すことができれば、更なるアンカーの発話区間検出精度、音声認識精度の向上が見込めると考えた。\par
本研究では「発話の時間間隔」と「発話環境」を考慮して、前後の発話区間が同一話者の発話である可能性が高いとき発話区間を結合し、長い発話を擬似的に作成した。これによって短い発話のi-vectorの抽出精度向上を目指した。検証の結果、アンカーの発話区間検出精度が約6\%向上した。また、音声認識においては、アンカーの発話区間が既知の場合と未知の場合で音声認識実験を行った。その結果、アンカーの発話区間が既知の場合は音声認識精度の向上は確認できなかったが、未知の場合は約9\%向上した。このように、発話区間を結合して、i-vectorを抽出することはアンカーの発話区間検出においては有意な結果を示した。音声認識においてはアンカーの発話区間が既知の場合効果を確認することができなかったが、原因として、ニュース番組内の背景雑音や音楽の存在がある。よって、今後、雑音除去を行ったうえで音声認識精度の検証を行う必要があると考えられる。