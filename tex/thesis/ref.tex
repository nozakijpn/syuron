\begin{thebibliography}{99}     %文献数が10未満の時 {9}
\bibitem{panaiv}辻川美沙貴,西川剛樹,松井知子,"i-vectorによる短い発話の話者識別の検討",電子情報通信学会(2015)
\bibitem{shimae_9}冨久祐介,“音源識別のための音クラスタリングとガウス分布混合数の有効性の検討”,
長崎大学工学部情報システム工学科平成,19年度卒業論文(2008)
\bibitem{iv}N.Dehak, P.Kenny, R.Dehak, P.Dumouchel and P.Ouellet,"Front-end factoranalysis forspeaker verifiCatiOn" IEEE Trans. Audio Speech Lang. Process, 19, 788-798(2011)
\bibitem{shimae_10}水野理, 大附克年, 松永昭一, 林良彦: “ニュースコンテンツにおける音響信号自動判
別の検討”, 電気情報通信学会総合大会(2003)
\bibitem{sp_recognition_shikano}:鹿野清宏,伊藤克亘,河原達也,武田一哉,山本幹雄,"音声認識システム",情報処理学会,オーム社(2003)
\bibitem{shimae_11}新美康永,“音声認識”,共立出版株式会社(1979)
\bibitem{ATR}国立情報学研究所データセット集合利用研究開発センター"ATRバランス文"
\bibitem{alize}"ALIZE",http:/alize.univ-avignon.fr
\bibitem{nozaki_gakuseikai}野崎大智,"ニュース音声におけるi-vectorを用いた同一アンカーの発話群の検出",電気情報通信学会九州支部学生会講演会(2018)
\bibitem{yoshimura_clustering}吉村竜哉,"話者クラスタ音響モデルを用いた会議音声認識のための話者適応",電気情報通信学会九州支部学生会講演会(2014)
\bibitem{kojima}小島和也,"会議音声認識のためのDNNを用いた高精度な音響モデルの構築法の検討",長崎大学工学部情報システム工学科 平成25年度修士論文(2013)
\bibitem{kaldi}"KALDI",http://kaldi.sourceforge.net/
\end{thebibliography}
