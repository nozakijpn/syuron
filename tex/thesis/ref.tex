\begin{thebibliography}{99}     %文献数が10未満の時 {9}
\bibitem{vad}Ramírez, J., J. M. Górriz, J. C. Segura,"Voice Activity Detection. Fundamentals and Speech Recognition System Robustness",pp. 1–22. ISBN 978-3-902613-08-0.(2007)
\bibitem{yamaguchi_indexing}山口 正秀,松永昭一,山下優,"Spectral Cross-Correlation Features for Audio Indexing of Broadcast News and Meetings",Eurospeech,pp613-615(2005)
\bibitem{yoshimura_clustering}吉村竜哉,"話者クラスタ音響モデルを用いた会議音声認識のための話者適応",電気情報通信学会九州支部学生会講演会(2014)
\bibitem{iv}N.Dehak, P.Kenny, R.Dehak, P.Dumouchel and P.Ouellet,"Front-end factoranalysis forspeaker verifiCatiOn" IEEE Trans. Audio Speech Lang. Process, 19, 788-798(2011)
\bibitem{nozaki_gakuseikai}野崎大智,"ニュース音声におけるi-vectorを用いた同一アンカーの発話群の検出",電気情報通信学会九州支部学生会講演会(2018)
\bibitem{adachi_gakuseikai}安達大輔,"ニュース番組における発話者群の段階的分類の検討",電気情報通信学会九州支部学生会講演会(2012)
\bibitem{panaiv}辻川美沙貴,西川剛樹,松井知子,"i-vectorによる短い発話の話者識別の検討",電子情報通信学会(2015)
\bibitem{ATR}国立情報学研究所データセット集合利用研究開発センター"ATRバランス文"
\bibitem{alize}"ALIZE",http:/alize.univ-avignon.fr
\bibitem{shimae_9}冨久祐介,“音源識別のための音クラスタリングとガウス分布混合数の有効性の検討”,
長崎大学工学部情報システム工学科平成,19年度卒業論文(2008)
\bibitem{shimae_10}水野理, 大附克年, 松永昭一, 林良彦: “ニュースコンテンツにおける音響信号自動判
別の検討”, 電気情報通信学会総合大会(2003)
\bibitem{sp_recognition_shikano}:鹿野清宏,伊藤克亘,河原達也,武田一哉,山本幹雄,"音声認識システム",情報処理学会,オーム社(2003)
\bibitem{audio_textbook}河原達也,"音声認識システム",情報処理学会(2016)
\bibitem{kaldi}"KALDI",http://kaldi.sourceforge.net/
\bibitem{kojima}小島和也,"会議音声認識のためのDNNを用いた高精度な音響モデルの構築法の検討",長崎大学工学部情報システム工学科 平成25年度修士論文(2013)
\bibitem{egashira_text}江頭一茂,"会議の発話分の特徴に合わせた会議音声認識用言語モデルの構築法の検討",長崎大学工学部情報システム工学科,平成25年修士論文(2013)
\bibitem{arai_text}荒井勇吏,"会議音声認識のための発話行為依存言語モデルの構築",長崎大学工学部情報システム工学科,平成26年修士論文(2014)
\bibitem{shimae_11}新美康永,“音声認識”,共立出版株式会社(1979)
\end{thebibliography}
