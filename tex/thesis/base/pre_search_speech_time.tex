\section{音源分離精度の調査}
\subsection{使用する音声データ}
\label{section:detail_train_news}
本調査では、ニュース番組の音声データ12個を用いる。各音声データには、事前に人手で3種類(音楽、音声、雑音)の音源ラベルが付与されている。「音声」の音源ラベルが付与された区間においては、更に発話者の情報が付与されている。表\ref{fig:example_label}は音声の音源ラベルの一例である。また「音声」の音源ラベルをもとに対象の音声データから発話区間を抽出し、それを一発話とした。表\ref{table:train_detail}に調査に用いるデータの詳細を示す。\vspace{0.2in}

\begin{table}[H]
\begin{center}
\caption{「音声」の音源ラベルの例 \label{fig:example_label}}
\begin{tabular}{|c|c|}
\hline
time      & speaker          \\ \hline
18.526910 & -1 male1\_INT\_S \\ \hline
20.793192 & -1 male1\_INT\_E \\ \hline
21.293665 & -1 male1\_INT\_S \\ \hline
23.116141 & -1 male1\_INT\_E \\ \hline
23.654385 & -1 male1\_INT\_S \\ \hline
26.270058 & -1 male1\_INT\_E \\ \hline
27.799800 & -1 male\_S       \\ \hline
29.811134 & -1 male\_E       \\ \hline
30.368265 & -1 male\_S       \\ \hline
34.277610 & -1 male\_E       \\ \hline
\end{tabular}
\end{center}
\end{table}

\begin{table}[H]
  \begin{center}
    \caption{調査音声データの詳細 \label{table:train_detail}}
    \begin{tabular}{|c||c|c|c|} \hline
      データID & 収録時間 & 話者数 & 全発話数 \\ \hline
      ニュースA & 30分3秒 & 20 & 337 \\ \hline
      ニュースB & 30分3秒 & 31 & 312\\ \hline
      ニュースC & 30分3秒 & 21 & 324 \\ \hline
      ニュースD & 30分4秒 & 20 & 324\\ \hline
      ニュースE & 20分3秒 & 13 & 159\\ \hline
      ニュースF & 30分3秒 & 22 & 343\\ \hline
      ニュースG & 30分4秒 & 22 & 313\\ \hline
      ニュースH & 30分4秒 & 20 & 315\\ \hline
      ニュースI & 30分4秒 & 17 & 321\\ \hline
      ニュースJ & 30分4秒 & 16 & 337\\ \hline
      ニュースK & 30分4秒 & 20 & 363\\ \hline
      ニュースL & 30分4秒 & 26 & 345\\ \hline
    \end{tabular}
  \end{center}
\end{table}

\subsection{調査方法}
\ref{section:devide_audio}節で述べた音源分離を用いての音源分離を行う。
表\ref{table:detail_identification_method1}は音源識別の調査条件である。

\begin{table}[H]
  \begin{center}
    \caption{音源識別実験の実験条件 \label{table:detail_identification_method1}}
    \begin{tabular}{|c||c|} \hline
      FFTの窓幅(フレーム長) & 2048point(約0.046[sec])   \\ \hline
      FFTのシフト幅(フレーム間隔) &  1024point(約0.023[sec]) \\ \hline
      窓関数 & ハミング窓  \\ \hline
    \end{tabular}
  \end{center}
\end{table}

また、検出する区間は「音声」「背景雑音」「音楽」「無音」の4つである。

\subsection{評価方法}
評価は、検出された各区間と正解ラベルを比較して行う。

\begin{table}[H]
\begin{center}
    \caption{検出した区間の正誤判定 \label{table:search_table}}
\begin{tabular}{|c|c|c|c|l}
\cline{1-4}
\multicolumn{2}{|c|}{\multirow{2}{*}{}} & \multicolumn{2}{c|}{正解ラベル} &  \\ \cline{3-4}
\multicolumn{2}{|c|}{}                  & ラベルが付与された区間        &    ラベルが付与されていない区間     &  \\ \cline{1-4}
\multirow{2}{*}{判定結果}        & 正        & $TP$                  & $FP$                   &  \\ \cline{2-4}
& 誤        & $FN$                  & $TN$                   &  \\ \cline{1-4}
\end{tabular}
\end{center}
\end{table}

表\ref{table:search_table}が得られると$P$(適合率(Precision))と$R$(再現率(Recall))は式\ref{calc:precision}と式\ref{calc:recall}のようにそれぞれ定義できる。

\begin{equation}
\label{calc:precision}
P = \frac{TP}{TP + FP}
\end{equation}

\begin{equation}
\label{calc:recall}
R = \frac{TP}{TP + FN}
\end{equation}

適合率が高い値を取るとき、識別結果に含まれる「誤り」の割合が少ないことを示している。また再現率が高いとき、識別結果に「漏れ」が少ないことを示している。一般的に、再現率の高いシステムは適合率が低く、逆に適合率が高いシステムは再現率が低い傾向にある。評価指標が2つあるとどちらのシステムが優れているかの判断が難しいため、適合率と再現率の調和平均を取り、ひとつのスカラ値に変換したF値(F-measure)がある。

\begin{equation}
\label{calc:fmeasure}
F = \frac{2 \times P \times R}{P + R}
\end{equation}

ここで$P$と$R$はそれぞれ適合率、再現率を表す。\par
本調査では、評価指標として適合率、再現率、F値を用いる。

\subsection{調査結果}
表\ref{table:NHK_speach_RPF} $\sim$ 表\ref{table:NHK_pause_RPF}に音源識別による識別精度を示す。
\begin{table}[H]
  \begin{center}
    \caption{発話区間検出精度 \label{table:NHK_speach_RPF}}
    \begin{tabular}{|c||c|c|c|} \hline
データID & Recall & Precision & F-meature \\ \hline
ニュースA & 0.892 & 0.966 & 0.928 \\ \hline
ニュースB & 0.888 & 0.963 & 0.924 \\ \hline
ニュースC & 0.883 & 0.963 & 0.921 \\ \hline
ニュースD & 0.902 & 0.952 & 0.927 \\ \hline
ニュースE & 0.884 & 0.970 & 0.925 \\ \hline
ニュースF & 0.907 & 0.974 & 0.939 \\ \hline
ニュースG & 0.907 & 0.961 & 0.933 \\ \hline
ニュースH & 0.843 & 0.966 & 0.900 \\ \hline
ニュースI & 0.886 & 0.982 & 0.932 \\ \hline
ニュースJ & 0.902 & 0.980 & 0.939 \\ \hline
ニュースK & 0.875 & 0.963 & 0.917 \\ \hline
ニュースL & 0.886 & 0.963 & 0.923 \\ \hline
 平均 & 0.888 & 0.967 & 0.926 \\ \hline
    \end{tabular}
  \end{center}
\end{table}

\begin{table}[H]
  \begin{center}
    \caption{音楽区間検出精度 \label{table:NHK_music_RPF}}
    \begin{tabular}{|c||c|c|c|} \hline
データID & Recall & Precision & F-meature \\ \hline
ニュースA & 0.467 & 0.565 & 0.511 \\ \hline
ニュースB & 0.508 & 0.640 & 0.566 \\ \hline
ニュースC & 0.507 & 0.687 & 0.583 \\ \hline
ニュースD & 0.429 & 0.661 & 0.520 \\ \hline
ニュースE & 0.481 & 0.633 & 0.547 \\ \hline
ニュースF & 0.627 & 0.699 & 0.661 \\ \hline
ニュースG & 0.611 & 0.936 & 0.740 \\ \hline
ニュースH & 0.570 & 0.406 & 0.474 \\ \hline
ニュースI & 0.481 & 0.648 & 0.552 \\ \hline
ニュースJ & 0.531 & 0.776 & 0.631 \\ \hline
ニュースK & 0.718 & 0.381 & 0.498 \\ \hline
ニュースL & 0.672 & 0.471 & 0.554 \\ \hline
 平均 & 0.537 & 0.622 & 0.576 \\ \hline
    \end{tabular}
  \end{center}
\end{table}

\begin{table}[H]
  \begin{center}
    \caption{背景雑音区間検出精度 \label{table:NHK_noise_RPF}}
    \begin{tabular}{|c||c|c|c|} \hline
データID & Recall & Precision & F-meature \\ \hline
ニュースA & 0.259 & 0.835 & 0.395 \\ \hline
ニュースB & 0.406 & 0.681 & 0.509 \\ \hline
ニュースC & 0.199 & 0.857 & 0.323 \\ \hline
ニュースD & 0.225 & 0.678 & 0.338 \\ \hline
ニュースE & 0.282 & 0.783 & 0.414 \\ \hline
ニュースF & 0.145 & 0.587 & 0.233 \\ \hline
ニュースG & 0.192 & 0.855 & 0.313 \\ \hline
ニュースH & 0.235 & 0.803 & 0.364 \\ \hline
ニュースI & 0.338 & 0.817 & 0.478 \\ \hline
ニュースJ & 0.268 & 0.746 & 0.395 \\ \hline
ニュースK & 0.268 & 0.906 & 0.413 \\ \hline
ニュースL & 0.349 & 0.511 & 0.415 \\ \hline
 平均 & 0.263 & 0.756 & 0.390 \\ \hline
    \end{tabular}
  \end{center}
\end{table}

\begin{table}[H]
  \begin{center}
    \caption{無音区間検出精度 \label{table:NHK_pause_RPF}}
    \begin{tabular}{|c||c|c|c|} \hline
データID & Recall & Precision & F-meature \\ \hline
ニュースA & 0.883 & 0.659 & 0.755 \\ \hline
ニュースB & 0.334 & 0.685 & 0.449 \\ \hline
ニュースC & 0.923 & 0.669 & 0.776 \\ \hline
ニュースD & 0.581 & 0.587 & 0.584 \\ \hline
ニュースE & 0.807 & 0.693 & 0.745 \\ \hline
ニュースF & 0.859 & 0.564 & 0.681 \\ \hline
ニュースG & 0.934 & 0.659 & 0.773 \\ \hline
ニュースH & 0.788 & 0.626 & 0.698 \\ \hline
ニュースI & 0.907 & 0.708 & 0.795 \\ \hline
ニュースJ & 0.763 & 0.645 & 0.699 \\ \hline
ニュースK & 0.887 & 0.615 & 0.726 \\ \hline
ニュースL & 0.602 & 0.702 & 0.648 \\ \hline
 平均 & 0.787 & 0.649 & 0.712 \\ \hline
    \end{tabular}
  \end{center}
\end{table}


音声区間の検出精度は高い精度を示した。背景雑音の区間はRecallが非常に低いがPrecisionが非常に高い結果となった。
また、ニュース番組によって発話区間の検出精度に差が生じた。

\subsection{考察}
背景雑音区間と音楽区間の検出精度が音声区間と無音区間の区間検出精度と比較して大きく下がっている。これは、背景雑音と音楽が音声区間と同時に存在していることが多いためである。背景雑音は街頭インタビュー中に存在することが多く、音楽はニュース番組のオープニング、またはエンディングとして存在する。このように、音声と背景雑音、もしくは音楽が同時に収録されていた場合、基本的に音声は背景雑音や音楽と比較して大きく編集されていることが多い。つまり、音声と背景雑音、あるいは音楽が同時に収録されている場合、音声区間として検出しているため、検出精度が低下したと考えられる。そのため、検出精度の向上のためには、複数音源の検出を同時に行うか、音源分離を行う必要があると考えられる。\par