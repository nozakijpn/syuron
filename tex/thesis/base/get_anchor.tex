\section{アンカーの発話群検出実験}
\subsection{i-vectorを用いたアンカーの発話区間抽出方法}
ニュース番組では、アンカー以外にインタビューイ(インタビューの受け手)や中継の有無によって話者数が大きく異なる。そのためクラスタ数を決定した場合、クラスタ数と話者数に不一致が起こり同一アンカーの発話群検出精度が低下する場合がある。そこで,同一話者の発話データのi-vectorはベクトル空間上で局所的に分布することに着目した。アンカーの発話数は非アンカーと比較して多いことから多くのアンカーの発話が局所的に集まると考えたため、同一アンカーの発話データをより精度よく検出できると考えた。\par
そこで、2つの発話データのi-vectorのコサイン類似度が閾値以上の場合、その2つの発話データの話者は同一話者であると仮定した。まず、全ての発話データ間のi-vectorのコサイン類似度を求める。次に、このコサイン類似度が閾値以上となる発話データ数が最も多い発話データを同一アンカーの発話データ群$O$のセントロイドとし、閾値以上(話者性が類似している)の全データをそのデータ群$O$の初期要素とする。\par
一方、i-vectorを抽出する発話データの発声の抑揚が大きい場合、同一話者の発話間のi-vectorであってもコサイン類似度が閾値以下になる場合がある。そこで、発話データ$u_i(\in O)$と発話データ群$O$の距離が一定距離以内であるとき、発話データ$u_i$は発話データ群$O$の要素として追加する。

\subsection{実験方法}
\subsection{評価方法}
評価は、正解ラベルを用いて検出された話者の発話区間と比較して行う。

\begin{table}[H]
\begin{center}
    \caption{アンカーの発話区間の正誤判定 \label{table:clustering}}
\begin{tabular}{|c|c|c|c|l}
\cline{1-4}
\multicolumn{2}{|c|}{\multirow{2}{*}{}} & \multicolumn{2}{c|}{「発話者」のラベルが付与された発話区間} &  \\ \cline{3-4}
\multicolumn{2}{|c|}{}                  & アンカーの発話区間        & アンカー以外の発話区間        &  \\ \cline{1-4}
\multirow{2}{*}{判定結果}        & 正        & $TP$                  & $FP$                   &  \\ \cline{2-4}
& 誤        & $FN$                  & $TN$                   &  \\ \cline{1-4}
\end{tabular}
\end{center}
\end{table}

表\ref{table:clustering}が得られると$P$(適合率(Precision))と$R$(再現率(Recall))は式\ref{calc:precision}と式\ref{calc:recall}のようにそれぞれ定義できる。

\begin{equation}
\label{calc:precision}
P = \frac{TP}{TP + FP}
\end{equation}

\begin{equation}
\label{calc:recall}
R = \frac{TP}{TP + FN}
\end{equation}

すなわち、適合率とは識別結果にどれだけ「ゴミ」がないかを表している。一方、再現率は識別にどれだけ「漏れ」がないかを表している。一方、したがって、適合率と再現率は大きい値ほど性能がよいことになる。ここで、2つのシステムを比較する場合は1次元のスカラ値によって、2値的な判断ができたほうが便利である。適合率と再現率をひとつのスカラ値に変換する手法として$F$値($F-measure$)がある。

\begin{equation}
\label{calc:fmeasure}
F = \frac{1}{\frac{1}{P} + \frac{1}{R}}
\end{equation}

ここで$P$と$R$はそれぞれ適合率、再現率を表す。本研究では、評価方法として適合率、再現率、$F$値を用いる。
\subsection{実験結果}
\subsection{考察}
