\section{アンカーの発話群検出実験}
本節では、i-vectorを用いてアンカーの発話区間検出を行う。
\label{chapter:get_anchor}
\subsection{実験方法}
本節では、\ref{chapter:connect_sp}節の手法1で結合した発話区間から再抽出したi-vectorを用いてアンカーの発話区間検出を行う。i-vectorを用いたアンカーの発話区間抽出は\ref{section:clustering}節の手法を用いる。$Th_{cos}$は0.8から1.5までの範囲を0.1刻みで検証を行う。また、Baselineとしてi-vectorの再抽出を行わずにアンカーの発話区間検出を行う。\par
i-vectorの抽出には、ALIZEとLIR RALを用いる。i-vectorの抽出に使用するUBMモデルの学習には読み上げ音声\cite{ATR}を使用する。読み上げ音声に収録されている各発話データからi-vectorを抽出する。発話データから抽出する音響特徴パラメータを表\ref{iv_feature2}に示す。また混合数は32とした。

\begin{table}[H]
  \begin{center}
    \caption{使用する音響特徴パラメータ \label{iv_feature2}}
    \begin{tabular}{|c||c|} \hline
      特徴量 & 次元数\\ \hline
      MFCC & 19  \\ 
      POW & 1  \\ 
      $\Delta$MFCC & 19 \\ 
      $\Delta$POW & 1 \\ 
      $\Delta\Delta$MFCC & 19 \\ 
      $\Delta\Delta$POW & 1 \\ \hline
      計 & 60 \\ \hline
    \end{tabular}
  \end{center}
\end{table}

\subsection{評価方法}
評価は、検出されたアンカーの発話区間と正解ラベルを比較して行う。

\begin{table}[H]
\begin{center}
    \caption{アンカーの発話区間の正誤判定 \label{table:clustering}}
\begin{tabular}{|c|c|c|c|l}
\cline{1-4}
\multicolumn{2}{|c|}{\multirow{2}{*}{}} & \multicolumn{2}{c|}{「発話者」のラベルが付与された発話区間} &  \\ \cline{3-4}
\multicolumn{2}{|c|}{}                  & アンカーの発話区間        & アンカー以外の発話区間        &  \\ \cline{1-4}
\multirow{2}{*}{判定結果}        & 正        & $TP$                  & $FP$                   &  \\ \cline{2-4}
& 誤        & $FN$                  & $TN$                   &  \\ \cline{1-4}
\end{tabular}
\end{center}
\end{table}

表\ref{table:clustering}を用いて、$P$(適合率(Precision))と$R$(再現率(Recall))を式\ref{calc:precision2}と式\ref{calc:recall2}のようにそれぞれ定義する。また、$F$値($F-measure$)を式\ref{calc:fmeasure2}のように定義する。

\begin{equation}
\label{calc:precision2}
P = \frac{TP}{TP + FP}
\end{equation}

\begin{equation}
\label{calc:recall2}
R = \frac{TP}{TP + FN}
\end{equation}

\begin{equation}
\label{calc:fmeasure2}
F = \frac{1}{\frac{1}{P} + \frac{1}{R}}
\end{equation}

ここで$P$と$R$はそれぞれ適合率、再現率を表す。

また、検出したアンカーの発話区間の割合を式\ref{calc:anchor_acc}のように定義して評価する。

\begin{equation}
\label{calc:anchor_acc}
Acc_{time} = \frac{検出したアンカーの発話区間の時間数}{アンカーの発話区間の時間数}
\end{equation}

本実験では、評価方法として適合率、再現率、$F$値、$Acc_{time}$を用いる。

\subsection{実験結果}
アンカーの発話区間検出精度を以下に示す。


\begin{table}[H]
  \begin{center}
    \caption{アンカーの発話区間検出精度(Baseline) \label{table:result_get_anchor_baseline}}
    \begin{tabular}{|c||c|c|c|c|} \hline
      $Th_{cos}$ & $Recall$ & $Precision$ & $F-measure$ & $Acc_{time}$\\ \hline
0.5 & 0.778 & 0.646 & 0.706 & 0.732 \\ \hline
0.6 & 0.649 & 0.765 & 0.702 & 0.637 \\ \hline
0.7 & 0.402 & 0.871 & 0.55 & 0.45 \\ \hline
0.8 & & & & \\ \hline
    \end{tabular}
  \end{center}
\end{table}

\begin{table}[H]
  \begin{center}
    \caption{アンカーの発話区間検出精度($Th_{time}=0.8)$ \label{table:result_get_anchor08}}
    \begin{tabular}{|c||c|c|c|c|} \hline
      $Th_{cos}$ & $Recall$ & $Precision$ & $F-measure$ & $Acc_{time}$\\ \hline
0.5 & 0.836 & 0.632 & 0.720 & 0.794 \\ \hline
0.6 & 0.812 & 0.768 & 0.789 & 0.774 \\ \hline
0.7 & 0.778 & 0.866 & 0.820 & 0.747 \\ \hline
0.8 & 0.656 & 0.920 & 0.766 & 0.655 \\ \hline

    \end{tabular}
  \end{center}
\end{table}

\begin{table}[H]
  \begin{center}
    \caption{アンカーの発話区間検出精度($Th_{time}=0.9$) \label{table:result_get_anchor09}}
    \begin{tabular}{|c||c|c|c|c|} \hline
      $Th_{cos}$ & $Recall$ & $Precision$ & $F-measure$ & $Acc_{time}$\\ \hline
0.5 & 0.839 & 0.636 & 0.723 & 0.794 \\ \hline
0.6 & 0.810 & 0.775 & 0.792 & 0.771 \\ \hline
0.7 & 0.782 & 0.865 & 0.821 & 0.747 \\ \hline
0.8 & 0.681 & 0.915 & 0.781 & 0.671 \\ \hline

    \end{tabular}
  \end{center}
\end{table}

\begin{table}[H]
  \begin{center}
    \caption{アンカーの発話区間検出精度($Th_{time}=1.0$) \label{table:result_get_anchor10}}
    \begin{tabular}{|c||c|c|c|c|} \hline
      $Th_{cos}$ & $Recall$ & $Precision$ & $F-measure$ & $Acc_{time}$\\ \hline
0.5 & 0.837 & 0.638 & 0.724 & 0.793 \\ \hline
0.6 & 0.811 & 0.764 & 0.787 & 0.768 \\ \hline
0.7 & 0.784 & 0.867 & 0.824 & 0.747 \\ \hline
0.8 & 0.683 & 0.912 & 0.781 & 0.668 \\ \hline

    \end{tabular}
  \end{center}
\end{table}

\begin{table}[H]
  \begin{center}
    \caption{アンカーの発話区間検出精度($Th_{time}=1.1$) \label{table:result_get_anchor11}}
    \begin{tabular}{|c||c|c|c|c|} \hline
      $Th_{cos}$ & $Recall$ & $Precision$ & $F-measure$ & $Acc_{time}$\\ \hline
0.5 & 0.840 & 0.609 & 0.706 & 0.793 \\ \hline
0.6 & 0.815 & 0.714 & 0.761 & 0.772 \\ \hline
0.7 & 0.773 & 0.871 & 0.819 & 0.740 \\ \hline
0.8 & 0.688 & 0.910 & 0.783 & 0.675 \\ \hline

    \end{tabular}
  \end{center}
\end{table}


\begin{table}[H]
  \begin{center}
    \caption{アンカーの発話区間検出精度($Th_{time}=1.2$) \label{table:result_get_anchor12}}
    \begin{tabular}{|c||c|c|c|c|} \hline
      $Th_{cos}$ & $Recall$ & $Precision$ & $F-measure$ & $Acc_{time}$\\ \hline
0.5 & 0.841 & 0.587 & 0.692 & 0.793 \\ \hline
0.6 & 0.811 & 0.734 & 0.771 & 0.768 \\ \hline
0.7 & 0.774 & 0.877 & 0.822 & 0.741 \\ \hline
0.8 & 0.687 & 0.907 & 0.782 & 0.673 \\ \hline

    \end{tabular}
  \end{center}
\end{table}

\begin{table}[H]
  \begin{center}
    \caption{アンカーの発話区間検出精度($Th_{time}=1.3$) \label{table:result_get_anchor13}}
    \begin{tabular}{|c||c|c|c|c|} \hline
      $Th_{cos}$ & $Recall$ & $Precision$ & $F-measure$ & $Acc_{time}$\\ \hline
0.5 & 0.841 & 0.589 & 0.693 & 0.793 \\ \hline
0.6 & 0.809 & 0.699 & 0.750 & 0.769 \\ \hline
0.7 & 0.776 & 0.868 & 0.819 & 0.741 \\ \hline
0.8 & 0.686 & 0.902 & 0.779 & 0.672 \\ \hline

    \end{tabular}
  \end{center}
\end{table}

\begin{table}[H]
  \begin{center}
    \caption{アンカーの発話区間検出精度($Th_{time}=1.4$) \label{table:result_get_anchor14}}
    \begin{tabular}{|c||c|c|c|c|} \hline
      $Th_{cos}$ & $Recall$ & $Precision$ & $F-measure$ & $Acc_{time}$\\ \hline
0.5 & 0.840 & 0.609 & 0.706 & 0.792 \\ \hline
0.6 & 0.793 & 0.714 & 0.751 & 0.753 \\ \hline
0.7 & 0.760 & 0.868 & 0.810 & 0.725 \\ \hline
0.8 & 0.679 & 0.917 & 0.781 & 0.667 \\ \hline

    \end{tabular}
  \end{center}
\end{table}

\begin{table}[H]
  \begin{center}
    \caption{アンカーの発話区間検出精度($Th_{time}=1.5$) \label{table:result_get_anchor15}}
    \begin{tabular}{|c||c|c|c|c|} \hline
      $Th_{cos}$ & $Recall$ & $Precision$ & $F-measure$ & $Acc_{time}$\\ \hline
0.5 & 0.835 & 0.590 & 0.691 & 0.786 \\ \hline
0.6 & 0.795 & 0.680 & 0.733 & 0.751 \\ \hline
0.7 & 0.735 & 0.856 & 0.791 & 0.701 \\ \hline
0.8 & 0.661 & 0.890 & 0.759 & 0.646 \\ \hline

    \end{tabular}
  \end{center}
\end{table}

実験の結果、発話区間を結合して再抽出したi-vectorを用いた手法が全体的に高い精度を示した。Baselineは$Th_{cos}$が0.8のときクラスタが小さくなりすぎたためアンカーの発話区間を検出出来なかった。また、Baselineは$Th_{cos}$が0.5のときが$F-measure$が最も高い値をとるのに対して、本提案手法では$Th_{cos}$が0.7のとき、$F-measure$が最も高い値をとった。\par
提案手法での発話区間検出精度の中では、$Th_{time}$が1.0かつ、$Th_{cos}$が0.7のとき、$F-measure$が最大となった。また、$Th_{cos}$が小さい時には$Recall$が高く、大きい時には$Precision$が高くなる傾向が全ての$Th_{time}$の閾値の場合で確認された。

\subsection{考察}
Baselineと再抽出したi-vectorを用いた手法を比較したとき、再抽出したi-vectorを用いた手法の方が$Th_{cos}$を高くした時に発話区間検出精度が向上している。このことから、i-vectorの抽出精度が向上したと考えられる。\par
再抽出したi-vectorを用いた手法の$Th_{time}$を変更した場合を比較した場合、明確な違いは見られなかった。
