\section{アンカーの発話群検出実験}
本節では、i-vectorを用いてアンカーの発話区間検出を行う。
\label{chapter:get_anchor}
\subsection{実験方法}
本節では、\ref{chapter:connect_sp}節の手法1で得られたi-vectorを用いてアンカーの発話区間検出を行う。i-vectorを用いたアンカーの発話区間抽出方法を\ref{section:clustering}節に示す。

\subsection{評価方法}
評価は、検出されたアンカーの発話区間と正解ラベルを比較して行う。

\begin{table}[H]
\begin{center}
    \caption{アンカーの発話区間の正誤判定 \label{table:clustering}}
\begin{tabular}{|c|c|c|c|l}
\cline{1-4}
\multicolumn{2}{|c|}{\multirow{2}{*}{}} & \multicolumn{2}{c|}{「発話者」のラベルが付与された発話区間} &  \\ \cline{3-4}
\multicolumn{2}{|c|}{}                  & アンカーの発話区間        & アンカー以外の発話区間        &  \\ \cline{1-4}
\multirow{2}{*}{判定結果}        & 正        & $TP$                  & $FP$                   &  \\ \cline{2-4}
& 誤        & $FN$                  & $TN$                   &  \\ \cline{1-4}
\end{tabular}
\end{center}
\end{table}

表\ref{table:clustering}が得られると$P$(適合率(Precision))と$R$(再現率(Recall))は式\ref{calc:precision}と式\ref{calc:recall}のようにそれぞれ定義できる。

\begin{equation}
\label{calc:precision}
P = \frac{TP}{TP + FP}
\end{equation}

\begin{equation}
\label{calc:recall}
R = \frac{TP}{TP + FN}
\end{equation}

すなわち、適合率とは識別結果にどれだけ「ゴミ」がないかを表している。一方、再現率は識別にどれだけ「漏れ」がないかを表している。一方、したがって、適合率と再現率は大きい値ほど性能がよいことになる。ここで、2つのシステムを比較する場合は1次元のスカラ値によって、2値的な判断ができたほうが便利である。適合率と再現率をひとつのスカラ値に変換する手法として$F$値($F-measure$)がある。

\begin{equation}
\label{calc:fmeasure}
F = \frac{1}{\frac{1}{P} + \frac{1}{R}}
\end{equation}

ここで$P$と$R$はそれぞれ適合率、再現率を表す。

また、検出したアンカーの発話区間の割合を式のように定義して評価する。

\begin{equation}
\label{calc:anchor_acc}
Acc_{time} = \frac{検出したアンカーの発話区間の時間数}{アンカーの発話区間の時間数}
\end{equation}

本研究では、評価方法として適合率、再現率、$F$値、$Acc_{time}$を用いる。

\subsection{実験結果}
アンカーの発話区間検出精度を以下に示す。

\begin{table}[H]
  \begin{center}
    \caption{アンカーの発話区間検出精度($Th_{time}=0.8)$ \label{table:result_get_anchor08}}
    \begin{tabular}{|c||c|c|c|c|} \hline
      $Th_{cos}$ & $Recall$ & $Precision$ & $F-measure$ & $Acc_{time}$\\ \hline
0.5 & 0.836 & 0.632 & 0.72 & 0.794 \\ \hline
0.6 & 0.812 & 0.768 & 0.789 & 0.774 \\ \hline
0.7 & 0.778 & 0.866 & 0.82 & 0.747 \\ \hline
0.8 & 0.656 & 0.92 & 0.766 & 0.655 \\ \hline

    \end{tabular}
  \end{center}
\end{table}

\begin{table}[H]
  \begin{center}
    \caption{アンカーの発話区間検出精度($Th_{time}=0.9$) \label{table:result_get_anchor09}}
    \begin{tabular}{|c||c|c|c|c|} \hline
      $Th_{cos}$ & $Recall$ & $Precision$ & $F-measure$ & $Acc_{time}$\\ \hline
0.5 & 0.839 & 0.636 & 0.723 & 0.794 \\ \hline
0.6 & 0.81 & 0.775 & 0.792 & 0.771 \\ \hline
0.7 & 0.782 & 0.865 & 0.821 & 0.747 \\ \hline
0.8 & 0.681 & 0.915 & 0.781 & 0.671 \\ \hline

    \end{tabular}
  \end{center}
\end{table}

\begin{table}[H]
  \begin{center}
    \caption{アンカーの発話区間検出精度($Th_{time}=1.0$) \label{table:result_get_anchor10}}
    \begin{tabular}{|c||c|c|c|c|} \hline
      $Th_{cos}$ & $Recall$ & $Precision$ & $F-measure$ & $Acc_{time}$\\ \hline
0.5 & 0.837 & 0.638 & 0.724 & 0.793 \\ \hline
0.6 & 0.811 & 0.764 & 0.787 & 0.768 \\ \hline
0.7 & 0.784 & 0.867 & 0.824 & 0.747 \\ \hline
0.8 & 0.683 & 0.912 & 0.781 & 0.668 \\ \hline

    \end{tabular}
  \end{center}
\end{table}

\begin{table}[H]
  \begin{center}
    \caption{アンカーの発話区間検出精度($Th_{time}=1.1$) \label{table:result_get_anchor11}}
    \begin{tabular}{|c||c|c|c|c|} \hline
      $Th_{cos}$ & $Recall$ & $Precision$ & $F-measure$ & $Acc_{time}$\\ \hline
0.5 & 0.84 & 0.609 & 0.706 & 0.793 \\ \hline
0.6 & 0.815 & 0.714 & 0.761 & 0.772 \\ \hline
0.7 & 0.773 & 0.871 & 0.819 & 0.74 \\ \hline
0.8 & 0.688 & 0.91 & 0.783 & 0.675 \\ \hline

    \end{tabular}
  \end{center}
\end{table}


\begin{table}[H]
  \begin{center}
    \caption{アンカーの発話区間検出精度($Th_{time}=1.2$) \label{table:result_get_anchor12}}
    \begin{tabular}{|c||c|c|c|c|} \hline
      $Th_{cos}$ & $Recall$ & $Precision$ & $F-measure$ & $Acc_{time}$\\ \hline
0.5 & 0.841 & 0.587 & 0.692 & 0.793 \\ \hline
0.6 & 0.811 & 0.734 & 0.771 & 0.768 \\ \hline
0.7 & 0.774 & 0.877 & 0.822 & 0.741 \\ \hline
0.8 & 0.687 & 0.907 & 0.782 & 0.673 \\ \hline

    \end{tabular}
  \end{center}
\end{table}

\begin{table}[H]
  \begin{center}
    \caption{アンカーの発話区間検出精度($Th_{time}=1.3$) \label{table:result_get_anchor13}}
    \begin{tabular}{|c||c|c|c|c|} \hline
      $Th_{cos}$ & $Recall$ & $Precision$ & $F-measure$ & $Acc_{time}$\\ \hline
0.5 & 0.841 & 0.589 & 0.693 & 0.793 \\ \hline
0.6 & 0.809 & 0.699 & 0.75 & 0.769 \\ \hline
0.7 & 0.776 & 0.868 & 0.819 & 0.741 \\ \hline
0.8 & 0.686 & 0.902 & 0.779 & 0.672 \\ \hline

    \end{tabular}
  \end{center}
\end{table}

\begin{table}[H]
  \begin{center}
    \caption{アンカーの発話区間検出精度($Th_{time}=1.4$) \label{table:result_get_anchor14}}
    \begin{tabular}{|c||c|c|c|c|} \hline
      $Th_{cos}$ & $Recall$ & $Precision$ & $F-measure$ & $Acc_{time}$\\ \hline


    \end{tabular}
  \end{center}
\end{table}

\begin{table}[H]
  \begin{center}
    \caption{アンカーの発話区間検出精度($Th_{time}=1.5$) \label{table:result_get_anchor15}}
    \begin{tabular}{|c||c|c|c|c|} \hline
      $Th_{cos}$ & $Recall$ & $Precision$ & $F-measure$ & $Acc_{time}$\\ \hline


    \end{tabular}
  \end{center}
\end{table}
\subsection{考察}
