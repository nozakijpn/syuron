\section{発話区間検出実験}

\subsection{実験条件}
\ref{section:devide_audio}節で述べた音源分離を用いて、発話区間検出を行う。表\ref{table:detail_identification_method}に音源識別の実験条件を示す。

\begin{table}[H]
  \begin{center}
    \caption{音源識別実験の実験条件 \label{table:detail_identification_method}}
    \begin{tabular}{|c||c|} \hline
      FFTの窓幅(フレーム長) & 2048point(約0.046[sec])   \\ \hline
      FFTのシフト幅(フレーム間隔) &  2048point(約0.023[sec]) \\ \hline
      窓関数 & ハミング窓  \\ \hline
    \end{tabular}
  \end{center}
\end{table}

\subsection{評価方法}
評価は、検出された区間と正解ラベルを比較して行う。

\begin{table}[H]
\begin{center}
    \caption{検出した区間の正誤判定 \label{table:search_table}}
\begin{tabular}{|c|c|c|c|l}
\cline{1-4}
\multicolumn{2}{|c|}{\multirow{2}{*}{}} & \multicolumn{2}{c|}{「音声」のラベルが付与された区間} &  \\ \cline{3-4}
\multicolumn{2}{|c|}{}                  & 発話区間        & 発話区間以外        &  \\ \cline{1-4}
\multirow{2}{*}{判定結果}        & 正        & $TP$                  & $FP$                   &  \\ \cline{2-4}
& 誤        & $FN$                  & $TN$                   &  \\ \cline{1-4}
\end{tabular}
\end{center}
\end{table}

表\ref{table:search_table}が得られると$P$(適合率(Precision))と$R$(再現率(Recall))は式\ref{calc:precision}と式\ref{calc:recall}のようにそれぞれ定義できる。

\begin{equation}
\label{calc:precision}
P = \frac{TP}{TP + FP}
\end{equation}

\begin{equation}
\label{calc:recall}
R = \frac{TP}{TP + FN}
\end{equation}

すなわち、適合率とは識別結果にどれだけ「ゴミ」がないかを表している。一方、再現率は識別にどれだけ「漏れ」がないかを表している。一方、したがって、適合率と再現率は大きい値ほど性能がよいことになる。ここで、2つのシステムを比較する場合は1次元のスカラ値によって、2値的な判断ができたほうが便利である。適合率と再現率をひとつのスカラ値に変換する手法としてF値(F-measure)がある。

\begin{equation}
\label{calc:fmeasure}
F = \frac{1}{\frac{1}{P} + \frac{1}{R}}
\end{equation}

ここで$P$と$R$はそれぞれ適合率、再現率を表す。\par
本実験では、評価方法として適合率、再現率、F値を用いる。
\subsection{実験結果}
表\ref{table:test_detail_RPF}に音源識別による発話区間検出精度を示す。

\begin{table}[H]
  \begin{center}
    \caption{発話区間検出精度 \label{table:test_detail_RPF}}
    \begin{tabular}{|c||c|c|c|} \hline
      データID & Recall & Precision & F-meature \\ \hline
      ニュース1 & 89.49 & 91.60 & 90.53 \\ \hline
      ニュース2 & 84.09 & 95.54 & 89.45\\ \hline
      ニュース3 & 88.30 & 85.99 & 87.13 \\ \hline
      ニュース4 & 90.06 & 83.33 & 86.56\\ \hline
      ニュース5 & 90.95 & 90.30 & 90.63\\ \hline
    \end{tabular}
  \end{center}
\end{table}

実験の結果、「音声」区間の検出精度は高い精度を示した。また、ニュース番組によって発話区間の検出精度に差が生じた。ニュース2はRecallが低く、ニュース3とニュース4ではPrecisionが低い結果となった。

\subsection{考察}
背景雑音区間の検出精度が大きく下がっているが、その理由として、人とシステムの感覚の相違が考えられる。雑音の定義上、人手でのラベル付けにおいて雑音の判断は難しい。例えば、背景雑音は無音との区別が難しい場合が多く、人がラベルで雑音とつけていてもシステムでは無音と識別する場合がよくある。ここに、システムを人間の感覚に近づける難しさがある。本実験で無音区間の検出精度が背景雑音区間の検出精度よりも比較的高いことも上記の問題によるものであると考えられる。\par
