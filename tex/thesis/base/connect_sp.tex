\section{発話区間の結合実験}
本節では、ニュース番組音声の発話区間を対象として前後の発話区間が同一話者である可能性が高いとき発話区間を結合し、i-vector抽出精度の向上を目指す。
\subsection{実験方法}
\ref{chapter:prob_method}章で述べた手法を用いて結合実験を行う。手法は以下の通りである。

\begin{itemize}
\item 手法1 : 発話間の時間情報を考慮した発話区間の結合手法
\item 手法2 : 発話環境を考慮した発話区間の結合手法
\item 手法3 : 手法1 + 手法2
\end{itemize}\par\par

また、手法1では非発話区間の長さの閾値によって結合するか否かを決定するため、閾値によって発話区間の結合精度が変化する。本実験では図\ref{fig:same_sp}より、0.8秒から1.5秒までを0.1秒刻みで閾値を変更して行う。また、非発話区間が閾値の時間より短いかつ、i-vectorのコサイン類似度が0.2以上の時、発話区間を結合するとした。

\subsection{評価方法}


\subsection{実験結果}
発話区間の結合精度の結果を表\ref{table:result_connect}に示す。

\begin{table}[htb]
  \begin{center}
    \caption{評価用音声データの発話区間検出精度$[\%]$ \label{table:result_connect}}
    \begin{tabular}{|c||c|c|c|} \hline
      データID & Recall & Precision & F-meature \\ \hline
      ニュース1 & 89.49 & 91.60 & 90.53 \\ \hline
      ニュース2 & 84.09 & 95.54 & 89.45\\ \hline
      ニュース3 & 88.30 & 85.99 & 87.13 \\ \hline
      ニュース4 & 90.06 & 83.33 & 86.56\\ \hline
      ニュース5 & 90.95 & 90.30 & 90.63\\ \hline
    \end{tabular}
  \end{center} 
\end{table}



\subsection{考察}
