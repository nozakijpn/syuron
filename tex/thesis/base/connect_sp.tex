\section{発話区間の結合実験}
本章では、ニュース番組音声の発話区間を対象として、前後の発話区間が同一話者である可能性が高いとき発話区間を結合する。
\subsection{実験条件}

\subsection{実験方法}
4章で述べた手法を用いて結合実験を行う。手法は以下の通りである。

\begin{itemize}
\item 手法1 : 発話間の時間情報を考慮した発話区間の結合手法
\item 手法2 : 発話環境を考慮した発話区間の結合手法
\item 手法3 : 手法1 + 手法2
\end{itemize}\par\par

\subsection{評価方法}

\subsection{実験結果}
発話区間の結合精度の結果を表\ref{table:result_connect}に示す。

\begin{table}[htb]
  \begin{center}
    \caption{評価用音声データの発話区間検出精度$[\%]$ \label{table:result_connect}}
    \begin{tabular}{|c||c|c|c|} \hline
      データID & Recall & Precision & F-meature \\ \hline
      ニュースA & 89.49 & 91.60 & 90.53 \\ \hline
      ニュースB & 84.09 & 95.54 & 89.45\\ \hline
      ニュースC & 88.30 & 85.99 & 87.13 \\ \hline
      ニュースD & 90.06 & 83.33 & 86.56\\ \hline
      ニュースE & 90.95 & 90.30 & 90.63\\ \hline
    \end{tabular}
  \end{center}
   
\end{table}

\subsection{考察}
