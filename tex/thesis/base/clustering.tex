ニュース番組では、アンカー以外にインタビューイ(インタビューの受け手)や中継の有無によって話者数が大きく異なる。そのためクラスタ数を決定した場合、クラスタ数と話者数に不一致が起こり同一アンカーの発話群検出精度が低下する場合がある。そこで,同一話者の発話データのi-vectorはベクトル空間上で局所的に分布することに着目した。アンカーの発話数は非アンカーと比較して多いことから多くのアンカーの発話が局所的に集まると考えたため、同一アンカーの発話データをより精度よく検出できると考えた。\par
そこで、2つの発話データのi-vectorのコサイン類似度が閾値以上の場合、その2つの発話データの話者は同一話者であると仮定した。まず、全ての発話データ間のi-vectorのコサイン類似度を求める。次に、このコサイン類似度が閾値以上となる発話データ数が最も多い発話データを同一アンカーの発話データ群$O$のセントロイドとし、閾値以上(話者性が類似している)の全データをそのデータ群$O$の初期要素とする。\par
一方、i-vectorを抽出する発話データの発声の抑揚が大きい場合、同一話者の発話間のi-vectorであってもコサイン類似度が閾値以下になる場合がある。そこで、発話データ$u_i(\in O)$と発話データ群$O$の距離が一定距離以内であるとき、発話データ$u_i$は発話データ群$O$の要素として追加する。
