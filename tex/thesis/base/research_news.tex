\chapter{ニュース番組音声における発話の間隔の調査}
本章では、ニュース番組音声の発話間隔の調査を行う。これは、同一話者が連続で発話する場合は次の発話までの間隔が非常に短いと考え、話者が切り替わる際は、映像の切り替わりなどがあるため発話の間隔が長いと考えた。\par
そのため、

\section{使用する音声データ}
パラメータの学習用にニュース番組の音声データ13個を用いる。各音声データには、事前に人手で4種類(音楽、音声、雑音、無音)の音源ラベルが付与されている。「音声」の音源ラベルが付与された区間においては、更に発話者の情報が付与されている。また「音声」の音源ラベルをもとに対象の音声データから発話区間を抽出し、それを一発話とした。\par
表\ref{table:train_detail}に検証に用いるデータの詳細を示す。\vspace{0.2in}

\begin{table}[htb]
  \begin{center}
  \label{table:train_detail}
    \caption{パラメータの学習用音声データの詳細}
    \begin{tabular}{|c||c|c|c|} \hline
      データID & 収録時間 & 話者数 & 全発話数 \\ \hline
      ニュースF & 30分3秒 & 20 & 337 \\ \hline
      ニュースG & 30分3秒 & 31 & 312\\ \hline
      ニュースH & 30分3秒 & 21 & 324 \\ \hline
      ニュースI & 30分4秒 & 20 & 324\\ \hline
      ニュースJ & 20分3秒 & 13 & 159\\ \hline
      ニュースK & 30分3秒 & 22 & 343\\ \hline
      ニュースL & 30分4秒 & 22 & 313\\ \hline
      ニュースM & 30分4秒 & 20 & 315\\ \hline
      ニュースN & 30分4秒 & 17 & 321\\ \hline
      ニュースO & 30分4秒 & 16 & 337\\ \hline
      ニュースP & 30分4秒 & 20 & 363\\ \hline
      ニュースQ & 30分4秒 & 26 & 345\\ \hline
      ニュースR & 30分4秒 & 26 & 314\\ \hline
    \end{tabular}
  \end{center}
\end{table}
\section{調査結果}