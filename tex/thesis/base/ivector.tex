近年の話者認識システムの多くは i-vector\cite{iv}に基づいて構成されおり、この領域における最高水準の技術となっている。i-vectorとは、ある発話から得られた音響特徴量を因子分析を用いて、話者固有の特徴を抽出したものである。i-vectorの抽出においては、因子分析の入力として、発話毎にGMM(Gaussian Mixture Model)の平均ベクトルを結合したGMMスーパーベクトルを用いる。発話$u$から作成された GMM スーパーベクトル$M_u∈R^{CD_F}$は以下で定義される。

\begin{equation}
M_u=T w_u + m, \notag
\end{equation}

ここで$M_u$は大量の不特定話者の発話データから作成されるUBM(Universal Background Model)を事前情報として事後確率最大化(MAP)法により推定されたGMMを用いる。
また$m$はUBMから得られる話者及びチャネル非依存のGMMスーパーベクトルである。
$C$は GMM (UBM)の混合数,$D_F$は音響パラメータの次元数、$T∈R^{CD_F*D_r}$は低ランクの矩形行列$D_r \ll CD_F$で、全変動空間を張る基底ベクトルで構成される固有声行列である。$W_u \in R^{D_r}$は発話ごとに与えられる潜在変数であり、平均ベクトルが$0 \in R^{D_T}$で共分散行列行列が単位行列$I \in R^{D_T*D_T}$のガウス分布$N(w ; 0,I)$に従う。この$w$はtotal factor(全因子)と呼ばれ、各発話に対するi-vector である。つまり、i-vectorはGMM スーパーベクトル空間における平均的な話者(UBM の平均)から「差(を次元圧縮したもの)」として各話者を表現したものと言える。

\subsection{UBMに対するBaum-Welch統計}
準備として、UBMに対するBaum-Welch統計量を計算することから始める。
$O_u={o_1,o_2,o_3,\cdots,o_L}$、$o_t\in R^{D_F}$、を発話$u$から得られる$L$フレームの音響パラメータ系列$c=1,2,3\cdots,C$、をUBM (GMM) の混合要素を表す添え字、$\Omega=\left\{\pi_c,m_c,\sum_{c}\right\}_{c=1}^{C}$をUBMのパラメータ(混合重み、平均ベクトル、対角共分散行列)とする。このとき、発話$u$に対する0次、1次、2次のBaum-Welch統計量は、

\begin{align}
N_{u,c} &=\sum_{t=1}^{L}\gamma_t(t), \notag \\
F_{u,c} &=\sum_{t=1}^{L}\gamma_t(c)(o_t-m_c), \notag \\
S_{u,c} &=diag\left[\sum_{t=1}^L\gamma_c(c)(o_c-m_c)(o_t-m_c)^\prime\right] \notag ,
\end{align}



と書ける。ここで、$y_c(c)$は、$o_t$がUBMの$c$番目の要素分布から生成される事後確率

\begin{equation}
\gamma_c(c)=\displaystyle p(c\mid o_t,Ω)=\frac{\pi_cp(o_t\mid m_c,\sum c)}{\sum_{k=1}^C \pi_kp(o_t\mid m_k,\sum k)}, \notag
\end{equation}

である。更にこれらを用いて、
\begin{align}
  N_u &= \left(
    \begin{array}{cccc}
      N_{u,1},I_{D_F} & 0 & \ldots & 0 \\
      0 & 0 & \ldots & \vdots \\
      \vdots & \vdots & \ddots & \vdots \\
      0 & 0 & \ldots & N_{u,C},I_{D_F}
    \end{array}
  \right)
\text{  $\in R^{CD_F\times CD_F}$}, \notag \\
  F_u &= \left(
    \begin{array}{cccc}
      F_{u,1} \\
      F_{u,2} \\
      \vdots\\
      F_{u,C}
    \end{array}
  \right)
\text{  $\in R^{CD_F}$}, \notag \\
  S_u &= \left(
    \begin{array}{cccc}
      S_{u,1},0 & 0 & \ldots & 0 \\
      0 & S_{u,2} & \ldots & \vdots \\
      \vdots & \vdots & \ddots & \vdots \\
      0 & 0 & \ldots & S_{u,C}
    \end{array}
  \right)
\text{  $\in R^{CD_F\times CD_F}$} \notag ,
\end{align}

ここで、$I_{D_F}\in R^{D_F\times D_F}$である。

\subsection{全因子$w$の確率分布とi-vectorの抽出}
本節では、$w$に関する種々の確率分布を導出する。このとき、$w$の事後分布の導出過程においてi-vectorの具体的な計算方法を示す。

\begin{itemize}
\item 事前分布\par
$w$の事前分布は$p(w)$平均0、共分散行列を持つガウス分布であり、以下のように書ける。
\begin{equation}
%\label{iv1}
p(w)\propto \exp(-\frac{1}{2}w^{\prime}w). \notag
\end{equation}

\item 条件付き分布\par
$M_{u,c}$を混合要素に対する$M_u$の部分ベクトルとする。直感的には、$M_{u,c}$は発話$O_u$で学習したGMMの混合要素$c$に割り当てられた$O_c$の各フレームは、平均$M_{u,c}$、共分散行列$\sum_{c}$(UBMのまま)に従うと仮定する。すなわち、$w$の値で条件付けられた観測データ$O$の条件付き分布は

\begin{equation}
\label{iv2}
P(O_u|w_u)=\exp\left(\sum_{t=1}^{L}\sum_{c=1}^{C}\gamma_t(c)\log(2\pi )^{-\frac{D_F}{2}}\left|\Sigma_{c}\right|^{-\frac{1}{2}}-\frac{1}{2}\sum_{t=1}^{L}\sum_{c=1}^{C}\gamma_t(c)D(o_t;\theta_c) \right),
\end{equation}
のように書ける。ここで、

\begin{align}
D(o_t;\theta_t) &=(o_t-M_{u,c})^\prime \Sigma_{c}^{-1}(o_t-M_{u,c}), \\
M_{u,c} &=m_c+T_cw_u.
\end{align}
である。$T_c\in R^{D_F\times D_T}$は、混合要素$c$に対する$T$の部分行列である。式(\ref{iv2})のexpの内部をBaum-Welch統計量を用いて整理すると、

\begin{equation}
\begin{split}
\sum_{t=1}^{L}\sum_{c=1}^{C}\gamma_t(c)log(2\pi )^{-\frac{D_F}{2}}\left|\Sigma_{c}\right|^{-\frac{1}{2}}-\frac{1}{2}\sum_{t=1}^{L}\sum_{c=1}^{C}\gamma_t(c)D(o_t;\theta_c)=G_u^\Sigma+H_u^{\Sigma T}+\text{Const.}
\end{split}
\end{equation}

ここで、$G_u^\Sigma$及び$H_u^{\Sigma T}$は、
\begin{align}
G_u^\Sigma &=\sum_{c=1}^C\left[\frac{1}{2}N_{u,c}\log\left|\Sigma_c^{-1}\right|-\frac{1}{2}tr\left(\Sigma_c^{-1}S_{u,c}\right)\right], \\
H_u^{\Sigma T} &=w_u^\prime T^\prime \Sigma^{-1}F_u-\frac{1}{2}w_u^\prime T^\prime N_u\Sigma^{-1}Tw_u. \label{iv3}
\end{align}

\item 事後分布\par
$w$の事後分布は(\ref{iv2})〜(\ref{iv3})を用いると、

\begin{align}
%\begin{split}
p(w_u|O_u) &\propto p(O_u|w_u)p(w_u)\propto \exp(w_u^\prime T^\prime \Sigma^\prime Tw_u-\frac{1}{2}w_u^\prime w_u) \notag \\ 
 &\propto exp(w_u^\prime T^\prime \Sigma^\prime F_u-\frac{1}{2}w_u^\prime N_u\Sigma^{-1}Tw_u-\frac{1}{2}w_u^\prime w_u) \notag \\ 
 &\propto \exp(-\frac{1}{2}(w_u-G_uT^\prime\Sigma^{-1}F_u)^\prime G_u^{-1}(w_u-G_uT^\prime \Sigma^\prime F_u)). \notag
%\end{split}
\end{align}
と書ける。ここで、

\begin{equation}
G_u=(I+T^\prime \Sigma^{-1}N_u T)^{-1}. \notag
\end{equation}

である。$w$の事後分布もガウス分布であることに注意すると、平均及び分散は、

\begin{align}
\label{iv4}
E[w_u] &=G_uT^\prime \Sigma^{-1}F_u, \\
cov[w_u] &=G_u. \notag
\end{align}

となる。前述のとおり、確率的潜在変数モデルのもと、i-vectorは$w$の事後分布の平均として得られる。つまり、発話$u$のi-vectorは、Baum-Welch統計量$N_u$、$F_u$及び推定済みのパラメータ$T$,$\Sigma$を用いて、式(\ref{iv4})により計算することができる。

\end{itemize}


\subsection{因子分析モデルパラメータの推定}
因子分析モデルのパラメータ$T$及び$\Sigma$は、EMアルゴリズムにより求められる。すなわち、完全データ${(O_u,w_u)}_{u=1}^{U}$に対する対数尤度の期待値

\begin{equation}
\label{iv5}
Q=\sum_{u=1}^{U}E[log p(O_ww_u|\theta)],
\end{equation}

の最大化問題を解くことで求める。ここで、$\theta$はパラメータ$T$、$\Sigma$を表す。完全データの対数尤度は、

\begin{equation}
\log p(O_w w_u)=logp(O_u|w_u,\theta)+logp(w_u), \notag
\end{equation}

と書けるので、式 (\ref{iv2})〜(\ref{iv3}) を用いると、式(\ref{iv5})は以下のように整理できる。

%\begin{equation}
%\label{iv6}
%\begin{split}
\begin{align}
Q=&\frac{1}{2}\sum_{u=1}^{U}\sum_{c=1}^{C}(N_{u,c} \log\left|\Sigma_c^{-1}\right|-tr(\Sigma_c^{-1}S_{u,c})), \notag  \\
&+\sum_{u=1}^{U}tr\left( \Sigma^{-1}\left( F_uE[w_u^\prime]T^\prime -\frac{1}{2}N_uTE[w_uw_u^\prime]T^\prime \right) \right) -\sum_{u=1}^{U}\frac{1}{2}tr(E[w_uW_u^\prime]). \label{iv6}
\end{align}
%\end{split}
%\end{equation}

以上より、Eステップにおいては古いパラメータを使って、$w$空間の事後分布の統計量を以下のように計算する。

\begin{align}
E[w_u] &=G_uT^\prime\Sigma^{-1}F_u, \notag \\
E[w_uw_u^\prime] &=G_u+E[w_u]E[w_u^\prime]. \notag
\end{align}

Mステップでは、式(\ref{iv6})をパラメータに関して最大化する。まず、(\ref{iv6})を$T$に関して微分して0と置くことで、以下の関係式を得る。

\begin{equation}
\sum_{u=1}^{U}\Sigma^{-1}F_uE[w_u^\prime]=\sum_{u=1}^{U}\Sigma^{-1}N_uTE[w_uw_u^\prime]. \notag
\end{equation}

これより、$T$の推定式が、

\begin{equation}
T^i=\left(\sum_{u=1}^{U}F_u^iE[w_u^\prime] \right)\left(\sum_{u=1}^{U}N_{u,c}E[w_uw_u^\prime] \right)^{-1}. \notag
\end{equation}

のように得られる。ここで、$T^i$、$F_u^i$は、おのおの$T$、$F_u$の$i$行目を表し、$i=(c-1)\times D_F+f$、$1\leq f\leq D_F$である。また、$\Sigma$の推定式は、

\begin{equation}
\Sigma=N^{-1}\left(\sum_{u=1}^{U}S_u-diag\left[\sum_{u=1}^{U}F_uE[w_u^\prime]T^\prime \right] \right). \notag
\end{equation}
となる。ここで、$N=\Sigma_{u=1}^{U}N_u$である。

\subsection{コサイン類似度}
発話$x$から抽出したi-vector$w_x$と発話$y$から抽出したi-vector$w_y$の比較を行うための方法としてコサイン類似度を用いる。

\begin{equation}
\cos(w_x,w_y)=\frac{w_x\cdot w_y}{\parallel w_x\parallel\parallel w_y\parallel}. \notag
\end{equation}

類似度の値の範囲は、$-1\leq \cos(w_x,w_y)\leq 1$であり、類似度が最も高い値は1である。

