近年の話者認識システムの多くは i-vector [3][4]に基づいて構成されおり、この領域における最高水準の技術となっている。i-vector とは、ある発話から得られた音響特徴量を因子分析を用いて、話者固有の特徴を抽出したものである。
 i-vector の抽出においては、因子分析の入力として、発話毎に GMM(Gaussian Mixture Model) の平均ベクトルを結合した GMM スーパーベクトルを用いる。発話$u$から作成された GMM スーパーベクトル$M_u∈R^{CD_F}$は以下で定義される。

\begin{equation}
M_u=T w_u + m
\end{equation}

ここで$M_u$は大量の不特定話者の発話データから作成されるUBM(Universal Background Model)を事前情報として事後確率最大化(MAP)法により推定されたGMMを用いる。
また$m$はUBMから得られる話者及びチャネル非依存のGMMスーパーベクトルである。
$C$は GMM (UBM)の混合数,$D_F$は音響パラメータの次元数、$T∈R^{CD_F*D_r}$は低ランクの矩形行列$D_r \ll CD_F$で、全変動空間を張る基底ベクトルで構成される固有声行列である。$W_u \in R^{D_r}$は発話ごとに与えられる潜在変数であり、平均ベクトルが$0 \in R^{D_T}$で共分散行列行列が単位行列$I \in R^{D_T*D_T}$のガウス分布N(w ; 0,I)に従う。この$w$はtotal factor(全因子)と呼ばれ、各発話に対するi-vector である。つまり、i-vectorはGMM スーパーベクトル空間における平均的な話者(UBM の平均)から「差(を次元圧縮したもの)」として各話者を表現したものと言える。

\subsection{UBMに対するBaum-Welch統計量}
すすす
\subsection{全因子$w$の確率分布とi-vectorの抽出}
あ
\subsection{因子分析モデルパラメータの推定}
あ
\subsection{コサイン類似度}
あ
